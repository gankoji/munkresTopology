% Created 2020-02-04 Tue 21:25
% Intended LaTeX compiler: pdflatex
\documentclass[11pt]{article}
\usepackage[utf8]{inputenc}
\usepackage[T1]{fontenc}
\usepackage{graphicx}
\usepackage{grffile}
\usepackage{longtable}
\usepackage{wrapfig}
\usepackage{rotating}
\usepackage[normalem]{ulem}
\usepackage{amsmath}
\usepackage{textcomp}
\usepackage{amssymb}
\usepackage{capt-of}
\usepackage{hyperref}
\usepackage{amsthm}
\newtheorem{theorem}{Theorem}[section]
\newtheorem{lemma}{Lemma}[section]
\newtheorem{definition}{Definition}[section]
\newtheorem{question}{Question}[section]
\newtheorem{summary}{Summary}[section]
\author{Jake Bailey}
\date{\today}
\title{Math 532 Notes}
\hypersetup{
 pdfauthor={Jake Bailey},
 pdftitle={Math 532 Notes},
 pdfkeywords={},
 pdfsubject={},
 pdfcreator={Emacs 26.3 (Org mode 9.3.1)}, 
 pdflang={English}}
\begin{document}

\maketitle
\tableofcontents


\section{Bases again \textit{<2020-01-22 Wed>}}
\label{sec:orgeed0654}

An eloquent summary: Bases must span the underlying space, and
intersections of elements must contain other elements (i.e. must have
sufficent granularity to contain every element of the set).

Open intervals predate topologies. 

\subsection{Example}
\label{sec:org639412b}
Let \(B\) be the collection fo all open intervals on \(\mathbb{R}\),
together with all sets of the form \((a,b)\setminus K\), where \(K = \{
1/n\ |\ n \in \mathbb{N}\}\). This is a basis for a topology on
\(\mathbb{R}\), called the K-topology, and on \(\mathbb{R}\) is also
written \(\mathbb{R}_K\).

\subsection{Back to bas(is)ics}
\label{sec:orgb4c069d}
\begin{lemma}
\(\mathbb{R}_l\) and \(\mathbb{R}_K\) are both finer than
\(\mathbb{R}_{std}\), but they are not comparable to each other.
\end{lemma}

On \(\mathbb{R}^2\) the collection of all open discs is a basis for a
topology, as are open balls in \(\mathbb{R}^n\). Open rectangles (cubes)
are also a basis, for the same topology. Proved by inclusion of open
sets.

\begin{question}
Prove that \(B_{disc}\) and \(B_{rect}\) generate the same topology. 
\end{question}

\begin{lemma}
Let \((X,T)\) be a topological space. Suppose \(C\) is a collection of
open sets such that \(\forall U \subseteq X\), and each \(x \in U\),
\(\exists c \in C\), such that \(x\in c \subseteq U\). Then \(C\) is a basis
for \(T\).
\end{lemma}

\begin{proof}
To be a basis, \(C\) a) Must cover \(X\), and b) if \(x \in c_1\cap c_2\),
then \(\exists c_3\) such that \(x \in c_3, c_3 \subseteq c_q\cap c_2\).

claim \(\forall x \in X, \exists c \in C\) such that \(x \in c\). Let \(x
\in X\). \$X is open in \(T\), so by construction of \(C\), \(\exists c \in
C\) such that \(x \in c \subseteq X\). 

Next, claim \(\forall c_1, c_2 \in C\) and \(\forall x \in c_1\cap c_2\),
\(\exists c_3 \in C\) such that \(x \in c_3 \subseteq c_1\cap c_2\). Let
\(x \in c_1\cap c_2\), \(c_1, c_2 \in C\). \(C\) is a collection of open
sets, i.e. members of \(T\). \(T\) is closed under finite intersections,
since it is a topology. Thus \(c_1\cap c_2 \in T\). By construction of
\(C\), \(\exists c_3 \in C\) such that \(x \in c_3 \subseteq c_1\cap c_2\). 

Let \(U\in T\). \(\forall x \in U\), \(\exists c_x \in C\) such that \(x \in
c_x \subseteq U\). Then, \(U = \bigcup\limits_{x \in U} c_x\). \(T'\) is
closed under unions, so \(U \in T'\).

Next, let \(U \in T'\). Then \(U = \bigcup\limits_{\alpha \in J}
c_{\alpha}\). Each \(c_{\alpha}\) is open (in \(T\)), by
construction. Thus, \(U \in T\). Hence, \(C\) is a basis, and it generates
\(T\).
\end{proof}

\subsection{The Dictionary Order Topology}
\label{sec:org49b527e}
\begin{definition}
A relation \(c\) on a set \(X\) is called an \textbf{order relation} if it
has the following properties:

\begin{enumerate}
\item Comparability: \(\forall\) x\textsubscript{1}, x\textsubscript{2}, either \(x_1 c x_2\), or \(x_2 c
   x_1\).
\item Non-reflexivity: For no \(x \in X\) does \(x c x\) hold.
\item Transitivity: \(\forall x,y,z \in X\), if \(x c y\) and \(y c z\), then
\(x c z\).
\end{enumerate}
\end{definition}
\end{document}