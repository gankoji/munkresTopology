% Created 2020-02-03 Mon 14:39
% Intended LaTeX compiler: pdflatex
\documentclass[11pt]{article}
\usepackage[utf8]{inputenc}
\usepackage[T1]{fontenc}
\usepackage{graphicx}
\usepackage{grffile}
\usepackage{longtable}
\usepackage{wrapfig}
\usepackage{rotating}
\usepackage[normalem]{ulem}
\usepackage{amsmath}
\usepackage{textcomp}
\usepackage{amssymb}
\usepackage{capt-of}
\usepackage{hyperref}
\usepackage{amsthm}
\newtheorem{theorem}{Theorem}[section]
\newtheorem{definition}{Definition}[section]
\author{Jake Bailey}
\date{\today}
\title{Math 532 Notes}
\hypersetup{
 pdfauthor={Jake Bailey},
 pdftitle={Math 532 Notes},
 pdfkeywords={},
 pdfsubject={},
 pdfcreator={Emacs 26.3 (Org mode 9.3.1)}, 
 pdflang={English}}
\begin{document}

\maketitle
\tableofcontents

\section{Lecture 7 \textit{<2020-01-31 Fri>}}
\label{sec:orgd0974d2}

The order topology vs subspace, continued

From examples, we know that the product and subspace topologies
"commute": the subspace topology on a product space is the same as the
product topology on a subspace.

The order topology, however, is a different story. It does not commute
with other topology combinations. Starting to see these as operations
on collections of sets. 

Idea: Let \(X\) be an ordered set \(Y\subseteq X\). The order on \(X\)
restricted to \(Y\) makes \(Y\) an ordered set. However, the resulting
order topology on \(Y\) may \textbf{not} be the same as the topology that \(Y\)
inherits as a subspace of \(X\). 

\subsection{Example 1}
\label{sec:orgcdf5c54}
\(Y = [0,1)\cup\{2\} \subseteq \mathbb{R}_{std}\).  

\(\{2\} = Y\cap(1,3)\) is open in the subspace topology.

But \(\{2\}\) is not open in the order topology on \(Y\).

\subsection{Example 2}
\label{sec:org9c5b435}
Let \(I = [0,1]\). Dictionary order on \(I\times I\) is the restriction of
the dictionary order on \(\mathbb{R}\times\mathbb{R}\). But, the
dictionary order topology is not the same as the subspace topology on
\(I\times I\). For example, consider the set
\(\{\frac{1}{2}\}\times(\frac{1}{2},1]\). This set is open in the
subspace topology, but \textbf{not} in the order topology on \(I\times
I\). This is because the subspace topology can create the closed
interval via intersection with the endpoints of I, but the order
topology does not contain half open intervals except for the
endpoints ("corners of the box"). 

\subsubsection{Key note}
\label{sec:orgce008d2}
In the dictionary order topology, half open intervals are only allowed
when they contain the \textbf{maximum} or \textbf{minimum} points of the underlying
set. That doesn't mean "the line on which the max x/y/z coordinate
lies", that mean the singular extrema (points).

\subsection{Convexity and Commutativity of Order and Subspace}
\label{sec:org82303d8}

\begin{definition}


Given an ordered set \(X\), \(Y\subseteq X\) is said to be convex if
\(\forall a,b \in Y\), we have that \((a,b) \subseteq Y\).
\end{definition}

\begin{theorem}


Let \(X\) be an ordered set in the order topology. Let \(Y\subseteq X\) be
convex. Then the order topology on \(Y\) is the same as the topology \(Y\)
inherits as a subspace of \(X\).
\end{theorem}
\begin{proof}


Let \(Y\) be a convex subset of \(Y\). Consider \((a, \infty) \subseteq
X\). Then \((a,\infty)\cap Y = \{x\ |\ x \in Y\ \text{and}\ x > a\}\) is an
open ray of \(Y\). Consider the case where \(a \not\in Y\). Then either
\(a\) is a lower bound of \(Y\), or \(a\) is an upper bound of \(Y\), since
\(Y\) is convex.

More succinctly, if \(a\) is lower bound, then \(a\cap Y = Y\), and if \(a\)
is an upper bound, \(a\cap Y = \emptyset\). Similarly, \((-\infty, a)\cap
Y = \emptyset\) (\(a\) is a lower bound), or \((-\infty, a)\cap Y = Y\)
(\(a\) is an upper bound).

We know that open rays form a subbasis for a topology. In this case,
they form a subbasis for the subspace topology on \(Y\). Each of these
is open in the order topology, so the order topology contains the
subspace topology (because the subbasis elements are open in the OT).

To prove the converse, note that any open ray of \(Y\) is the
intersection of an open ray of \(X\) and the set \(Y\). Since the open
rays of \(Y\) are a subbasis for the order topology on \(Y\) (see previous
paragraph), this topology is contained in the subspace topology.

The missing part of this proof is for the case of rays where \(a \in
Y\). If you intersect an open ray with \(Y\), no matter what, you get an
open set of \(Y\). My last question is, do the union of those two
raysets give you back \(Y\) if \(a \in Y\)? No, but you can choose all \(a\)
in \(Y\), because a subbasis does not have to be a partition (open sets
can overlap).
\end{proof}

\section{Closed sets and limit points}
\label{sec:org51304c4}
\textbf{Limit points} are my old nemesis from analysis. 

A set \(A \subseteq X\) is closed if \(X\setminus A\) is open.


\subsection{Example 1}
\label{sec:orgcf14560}

\([a,b] \subseteq \mathbb{R}\). \(\mathbb{R} \setminus [a,b] = (-\infty,a)\cup (b,\infty)\). 

\subsection{Example 2}
\label{sec:org2d8cea3}

$\backslash${ (x,y)$\backslash$ |$\backslash$ x\(\ge\) 0, y\(\ge\) 0$\backslash$}

\subsection{Finite Complement Topology}
\label{sec:org6e64dae}

On \(\mathbb{R}\), the finite complement topology takes as basis
elements all sets \(U\) such that \(\mathbb{R}\setminus U\) is finite (as
in finite number of points). Use DeMorgan's laws to prove this is a
topology. Think about whether this is a topology, or its complement.
\end{document}