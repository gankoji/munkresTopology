% Created 2020-02-03 Mon 15:34
% Intended LaTeX compiler: pdflatex
\documentclass[11pt]{article}
\usepackage[latin1]{inputenc}
\usepackage[T1]{fontenc}
\usepackage{graphicx}
\usepackage{grffile}
\usepackage{longtable}
\usepackage{wrapfig}
\usepackage{rotating}
\usepackage[normalem]{ulem}
\usepackage{amsmath}
\usepackage{textcomp}
\usepackage{amssymb}
\usepackage{capt-of}
\usepackage{hyperref}
\usepackage{amsthm}
\newtheorem{theorem}{Theorem}[section]
\newtheorem{lemma}{Lemma}[section]
\newtheorem{definition}{Definition}[section]
\newtheorem{question}{Question}[section]
\newtheorem{summary}{Summary}[section]
\author{Jake Bailey}
\date{\today}
\title{Math 532 Notes}
\hypersetup{
 pdfauthor={Jake Bailey},
 pdftitle={Math 532 Notes},
 pdfkeywords={},
 pdfsubject={},
 pdfcreator={Emacs 26.3 (Org mode 9.3.1)}, 
 pdflang={English}}
\begin{document}

\maketitle
\tableofcontents


\section{Lecture 1 \textit{<2020-01-15 Wed>}}
\label{sec:org630fe20}

Basic introductions to the course. Office hours M 2-3 (Math 220), TW
2-3 (Math 219), and by appointment. Book is Topology by Munkres. 

\subsection{Topological Spaces}
\label{sec:org7e13e86}

What are open sets in \(\mathbb{R}\)? We know they can be intervals,
like \((1,2)\) or \((a,b)\). \(\mathbb{R}\) itself is another example, as
well as its complement, \(\emptyset\).

\begin{question}
What questions does Topology answer. Which does it ask? 
\end{question}

\begin{question}
What is the origin of the name "Point-Set Topology"?
\end{question}

The set of all open intervals defines the "standard topology" on
\(\mathbb{R}\).

We'll also be concerned with homeomorphisms (Donuts + Coffee mugs). 

Much of "Point-Set Topology" is motivated by necessity from real analysis. 

Metrics induce a topology. A reasonable question in topology is "how
close is this space to being metrizable?"

\begin{summary}
We'll cover fundamental Point-Set Topology as well as som algebraic
topology in this course. We begin with open sets in \(\mathbb{R}\).
\end{summary}

\subsection{Class Outline}
\label{sec:orgf94402b}
\begin{itemize}
\item Basics of Topology
\item Waht is a topology
\item Continuous functions
\item Building new topologies
\begin{itemize}
\item Product, Subspace, Metric, Quotient T's
\end{itemize}
\item Connectedness
\item Compactness
\item Separation Axioms
\item Metrizability
\item Tychonoff's Theorem
\item Topics in Algebraic Topology
\begin{itemize}
\item Homotopy
\item Fundamental Groups
\item Boraz-Luanne?
\end{itemize}
\end{itemize}


\subsection{Back to Topology}
\label{sec:orgdf5e52d}
\begin{definition}


Given a set \(X\), a topology on \(X\) is a collection \(T\) of subsets of
\(X\) with the following properties:

\begin{enumerate}
\item \(\emptyset, X \in T\)
\item If \(X_{\alpha} \in T, \forall \alpha \in J\), then
\(\bigcup\limits_{\alpha \in J}X_{\alpha} \in T\). (\(J\) arbitrary
indices).
\item If \(J\) is finite, then \(\cap_{\alpha \in J}X_{\alpha} \in T\).
\end{enumerate}
\end{definition}

Topologies are collections of sets that 
\begin{enumerate}
\item Contain the empty set and the whole space
\item Are closed under arbitrary unions
\item Are closed under finite intersections
\end{enumerate}

\begin{question}


Where/how does an infinite intersection break a topology? Why are
finite intersections the only kind allowed? 
\end{question}
\begin{summary}


A topology is a collection of sets with three key properties
\end{summary}

\subsection{Topologies on \(\mathbb{R}\)}
\label{sec:org0b1ef43}
\begin{enumerate}
\item The standard topology, \(T_{std}\).
\item The lower limit topology, \(\mathbb{R}_l\)
\begin{enumerate}
\item Basis of elements as \([a,b)\)
\end{enumerate}
\item Any set \(X\) has two topologies
\begin{enumerate}
\item The discrete topology, \(T = 2^X\)
\item The indiscrete topology, \(T = \{\emptyset, X\}\)
\end{enumerate}
\end{enumerate}


\begin{question}


Are there sets for which the discrete and indiscrete topologies are
equivalent?
\end{question}

\subsection{Example}
\label{sec:orga81d598}
Take \(X = \{1,2,3\}\). Then, the following are topologies on \(X\): \(T =
\{\{1\},\{1,2\},X,\emptyset\}\), \(T =
\{\{2\},\{1,2\},\{2,3\},X,\emptyset\}\). The following is not: \(T =
\{\{1\},\{3\},X,\emptyset\}\).

\begin{summary}


Topologies can be formed on all sets, even ones we wouldn't categorize
with others, e.g. \(\mathbb{R}\) vs \(\mathbb{N}\), etc.
\end{summary}

\section{More on topological spaces \textit{<2020-01-17 Fri>}}
\label{sec:org21ef3e9}
\begin{definition}


Suppose that \(T, T'\) are two topologies on \(X\). If \(T \subseteq T'\),
we say that \(T'\) is \textit{finer} than \(T\), and that \(T\) is
\textit{coarser} than \(T'\). We say that \(T\) and \(T'\) are comparable if
either is contained in the other. 
\end{definition}

\begin{definition}


If \(X\) is a set, a \textit{basis} for a topology on \(X\) is a
collection \(B\) of subsets such that 
\begin{enumerate}
\item \(\forall x \in X, \exists b \in B\) such that \(x \in b\).
\item If \(b_1, b_2 \in B\), then \(\exists b_3 \in B\) such that \(b_3
   \subseteq b_1\cap b_2\).
\end{enumerate}
\end{definition}

\begin{question}


Is the standard topology in \(\mathbb{R}\) comparable with the lower
limit topology?
\end{question}

\begin{definition}


A set \(X\) along with a topology is called a \textit\{topological
space\}.
\end{definition}

\begin{definition}


If \((X,T)\) is a topological space, than a set \(Y\subseteq X\) is open
iff \(Y \in T\). \(Z\subseteq X\) is closed iff \(Z\setminus X\) is
open. (Open means a set is a member of the topology). 
\end{definition}

\subsection{Example}
\label{sec:org9cbfb6e}
\$(\mathbb{R}, 2\textsuperscript{\mathbb{R}}). 

All sets in \(\mathbb{R}\) are open, and they're also closed. 

\((\mathbb{R}, \{\emptyset, \mathbb{R}\})\).

Open sets are only \(\emptyset, \mathbb{R}\). These are also the only
closed sets. 

\((\mathbb{R}, T_{std})\)

\((0,5)\) is open. \(\mathbb{R}\setminus (0,5)\) is closed.

Recall that \(T_{std}\) is generated by the collection of open intervals
\((a,b)\) in \(\mathbb{R}\). 

\begin{question}


Do bases have to be minimal and/or orthonormal? Does that question
even make sense? 
\end{question}

\begin{question}


Can we characterize topologies where open sets are also closed?
\end{question}

\subsection{Bases}
\label{sec:org4d7ca16}

The topology generated by a basis \(B\) is defined as: \(U\) is open in
\(X\) if \(\forall x \in U, \exists b \in B\) such that \(x \in b \subseteq
U\).

To check is a set \(U\) is open, 

\begin{enumerate}
\item Pick \(x \in U\)
\item Find \(b\in B\) such that \(x \in b \subseteq U\).
\end{enumerate}



\subsubsection{Example}
\label{sec:org68b4ac4}
Is \(U = (0,5)\cup(15,20)\) open in \(\mathbb{R}_{std}\)? Yes. 

\begin{proof}


Let \(x \in U\). Then \(x \in (0,5)\) or \(x \in (15,20)\). Case 1: \(x \in
(0,5)\). \((0,5) \in B\), and \((0,5) \subseteq (0,5)\). Case 2 is
equivalent.
\end{proof}

Checking for openness is a \textit{pointwise} operation. 

\subsection{Generating Topologies from Bases}
\label{sec:org8c71044}
\begin{theorem}


Given a set \(X\) and a basis \(B\), the collection of sets \(T\) generated
by \(B\) as described above is a topology. 
\end{theorem}

\#+BEGIN\textsubscript{proof}

We look to show 3 properties:
\begin{enumerate}
\item \(\emptyset, X \in T\)
\item \(T\) is closed under unions
\item \(T\) is closed under finite intersections
\end{enumerate}



\begin{enumerate}
\item \(\emptyset \in T\) is vacuously true. For \(X \in T\), Choose \(U =
   X\). Then, \(\exists b\in B\) such that \(x \in b\). Satisfies clause 1.
\item Let \(U_{\alpha} \in T\). For any \(x \in \cup U_{\alpha}\), \(\exists b
   \in B\) such that \(x \in b, b \subseteq U_{\alpha}\), since \(\exists
   \alpha'\) such that \(x \in \cup U_{\alpha'}\), and \(U_{\alpha'}\) is
open.
\item For finite intersections: induction. Prove \(T\) is closed under
pairwise intersections, then assume \(n\) intersections, prove \(n +
   1\). Similar to unions, since \(x \in U = U_1\cap U_2\) implies \(x \in
   U_1\) and \(x \in U_2\), so \(\exists b \in B\) such that \(x\in n\), and
\(b \subseteq U_1\) and \(b \subseteq U_2\).
\end{enumerate}


\begin{lemma}


Let \(X\) be a set. Let \(B\) be a basis for a topology on \(X\). Then \(T\)
equals the collection of all unions of elements of \(B\).
\end{lemma}

Proof by subsetting both ways. 

\begin{lemma}


Let \((X,T)\) be a topological space. Suppose \(C\) is a collection of
open sets of \(X\) such that \(\forall U \subseteq X\), \(U\) open, and
\(\forall x \in U\), \(\exists c \in C\) such that \(x \in c\subseteq
U\). Then \(C\) is a basis for a topology on \(X\). 
\end{lemma}
\end{document}