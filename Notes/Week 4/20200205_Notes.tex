% Created 2020-02-07 Fri 09:48
% Intended LaTeX compiler: pdflatex
\documentclass[11pt]{article}
\usepackage[utf8]{inputenc}
\usepackage[T1]{fontenc}
\usepackage{graphicx}
\usepackage{grffile}
\usepackage{longtable}
\usepackage{wrapfig}
\usepackage{rotating}
\usepackage[normalem]{ulem}
\usepackage{amsmath}
\usepackage{textcomp}
\usepackage{amssymb}
\usepackage{capt-of}
\usepackage{hyperref}
\usepackage{amsthm}
\newtheorem{theorem}{Theorem}[section]
\newtheorem{lemma}{Lemma}[section]
\newtheorem{definition}{Definition}[section]
\newtheorem{question}{Question}[section]
\newtheorem{summary}{Summary}[section]
\newtheorem{corollary}{Corollary}[section]
\author{Jake Bailey}
\date{\today}
\title{Math 532 Notes}
\hypersetup{
 pdfauthor={Jake Bailey},
 pdftitle={Math 532 Notes},
 pdfkeywords={},
 pdfsubject={},
 pdfcreator={Emacs 26.3 (Org mode 9.3.3)}, 
 pdflang={English}}
\begin{document}

\maketitle
\tableofcontents


\section{Closed Sets and Limit Points \textit{<2020-02-05 Wed 09:04>}}
\label{sec:org9e64f2d}

Recall that a set \(A\) is closed iff its complement \(X\setminus A\) is
open.

\(int(A)\) [The interior of A] is the union of all open sets contained in \(A\). 

\(cl(A)\) [The closure of A] is the intersection of all closed sets
containing \(A\).

\begin{theorem}
Let \(Y\) be a subspace of \(X\), \(A \subseteq Y\). The closure of \(A\) in
\(Y\) equals \(cl(A) \cap Y\).
\end{theorem}

\begin{definition}
Two sets \(A,B\) \textbf{intersect} if their intersection is not empty,
i.e. \(A \cap B \not= \emptyset\).
\end{definition}

\begin{theorem}
Let \(A \subseteq X\). Then, \(x \in cl(A)\) iff every open set containing
\(x\) intersects \(A\). Also, if the topology of \(X\) is given by a basis,
\(x \in cl(A)\) iff every basis element of \(B\) containing \(x\) intersects \(A\).
\end{theorem}

WEIRD MOMENT: Munkres doesn't mention this, but we worked it through
in class. Need to add the requirement/assumption that \(x\in X\) in
order for this theorem to work.

\subsection{Example}
\label{sec:org169ef1e}

\(X = \mathbb{R}_{std}, A = (0,1]\). Then, \(cl(A) = [0,1]\).

\(B = \{ 1/n\ |\ n \in \mathbb{N}\}\). Then, \(cl(B) = \{0\}\cup B\).

\(cl(\mathbb{Q}) = \mathbb{R}\).

\(\overline{\mathbb{R}\setminus\mathbb{Q}} = \mathbb{R}\).

Let \(Y = (0,1] \subseteq \mathbb{R}\), \(A = (0, 1/2) \subseteq
Y\). Closure of \(A\) in \(\mathbb{R}\) is \([0,1/2]\). Closure in \(Y\) is
\(Y\cap \overline{A} = (0,1/2]\).

\subsubsection{Sidenote}
\label{sec:org8ecfc72}
I noticed this morning, as I was furiously trying to update my config
to get it up to speed for notetaking (since I goofed and didn't sync
my changes from the desktop last night), that the laptop has code
completion in elisp (and I'm suspecting, probably elsewhere). Diff the
configs and see what's different, since I'm missing that on both the
work and home pcs. 

\subsection{Limit points}
\label{sec:org8ee247d}
\begin{definition}
If \(X\) is a topological space, \(A\subseteq X\), \(x \in X\), then \(x\) is
a limit point of \(A\) if every open neighborhood of \(x\) intersects \(A\)
in some point other than \(x\) itself. 
\end{definition}

Limit points of \(A\) may not be in \(A\), but they also could be. It
tells us about the structure of \(A\) if it contains all of its limit
points.

Note: \(U\) is an open neighborhood of \(x\) iff \(U\) is open and \(x \in
U\).

\subsubsection{Examples of limit points}
\label{sec:org694e85e}
\(A = (0,1]\). Here, 0 is a limit point of \(A\). So is every point in
\(A\).

\(B= \{ 1/n\ |\ n\in\mathbb{N}\}\). 0 is a limit point of \(B\). Points of
\(B\) are not limit points. Note that the "open neighborhoods" of a
point should come from some underlying space, not just the set under
consideration. 0 is the only limit point. 

Note: \(x\) is a limit point of \(A\) iff \(x \in
\overline{A\setminus\{x\}}\).

\subsection{The ambitious theorem \textit{<2020-02-05 Wed 09:41>}}
\label{sec:orgb1cdd7a}
\begin{theorem}
Let \(A \subseteq X\), \(X\) a topological space. Let \(A'\) be the set of
limit points of \(A\). Then, \(\overline{A} = A\cup A'\).
\end{theorem}

\begin{proof}
Inclusion in both directions. 

As observed earlier, \(A \subseteq \overline{A}\). Also, by definition,
every limit point of \(A\) is in \(\overline{A}\), since a limit point has
its open neighborhoods intersect \(A\), and the closure points also have
open neighborhoods intersect \(A\). Thus, \(A\cup A' \subseteq \overline{A}\) is
obvious.

Suppose \(x \in \overline{A}\). If \(x \in A\), then done. Suppose \(x \ni
A\). \(x \in \overline{A}\), so every open neighborhood \(U\) of \(x\)
intersects \(A\). Let \(y \in U\cap A\). \(y\not = x\), since \(y \in A\). So,
\(x\) is a limit point, and thus \(x \in A'\). We've proven both cases, so
\(\overline{A} \subseteq A\cup A'\).
\end{proof}

If a point is in the closure of a set, it is either \textit\{in the set
itself\}, or it its \textit{a limit point} of it. There are no other
possibilities.

\section{Closed Sets Again, and maybe Hausdorff Spaces? \textit{<2020-02-07 Fri 09:00>}}
\label{sec:org5839cb4}
\begin{definition}
If \(X\) is a topological space, \(A\subseteq X\), \(x \in X\) then \(x\) is a
limit point of \(A\) if every neighborhood of \(x\) intersects \(A\) in a
point other than \(x\) itself.

I.e. \(\forall U\) open with \(x \in U\), \((U\setminus\{x\})\cap A
\not=\emptyset\).
\end{definition}

\begin{theorem}
Let \(A\subseteq X\), \(A'\) a set of limit points of \(A\). Then
\(\overline{A} = A\cup A'\).
\end{theorem}

\begin{corollary}
A subset \(A\) of a topological space is closed iff \(\overline{A} =
A\). (This holds iff \(\overline{A} \subseteq A\) iff \$ A' \subseteq A\$.
\end{corollary}

\subsection{Prepping for Hausdorff Spaces}
\label{sec:orgea3cdef}
Hausdorff property is important for convergence of sequences.

Consider a one point set in \(\mathbb{R}\), say
\(\{x_0\}\). \(\overline{\{x_0\}} = \{x_0\}\).

So one point sets are closed in \(\mathbb{R}\), but this is not true in
general. This is a convenient property of \(\mathbb{R}\).

Another convenient property of \(\mathbb{R}\), is that any sequence
which converges does so to only one point. In arbitrary topological
spaces, sequences may converge to more than one point. 

\begin{definition}
The traditional definition of convergence depends on a metric. A
sequence \(x_n\) converges to \(x\) if, \(\forall \epsilon \in \mathbb{R}\),
\(\exists N \in \mathbb{N}\) such that for \(n > N\), \(|x_n - x| <
\epsilon\).
\end{definition}

However, this definition doesn't do us any good if the space we're
considering doesn't have a metric. 

\begin{definition}
Topological definition of convergence. In a topological space \(X\) we
say that the sequence $\backslash${x\textsubscript{n}$\backslash$}\textsubscript{n \(\in\) \mathbb{N}} converges to \(x \in
X\) if for every open neighborhood \(U\) of \(x\), \(\exists N \in
\mathbb{N}\) such that \(\forall n > N\), \(x_n \in U\).
\end{definition}

\subsubsection{Example}
\label{sec:orgccfd63f}
Consider the topological space \((X, T)\), \(X = \{ a, b, c\}\), and \(T =
\{\emptyset, X, \{b\}, \{a, b\}, \{b, c\}\}\). Consider the sequence
\(\{x_n\}\) where \(x_n = b, \forall n\). Does this sequence converge to
\(b\)? Yes, there is always a neighborhood of \(b\) within which \(x_n\)
lies. It also converges to \(a\), since there is no open neighborhood of
\(a\) which does not contain \(b\) (same for \(c\)). 

\subsection{Introducting Hausdorff Spaces}
\label{sec:org6a4a75c}
\begin{definition}
A topological space \(X\) is called a Hausdorff space if, for any two
distinct points \(x_1, x_2 \in X\), there exist open neighborhoods \(U_1,
U_2\), of \(x_1, x_2\) respectively, such that \(U_1\cap U_2 = \emptyset\).
\end{definition}

This is one of the separation axioms? Why is it an axiom? Are we
\textbf{assuming} this to be true about \(\mathbb{R}\)? Is this
equivalent to the axiom of choice, or completeness? This condition is
stronger than the statement that finite point sets are closed (which
is also called the \(T_1\) axiom). 

Asked in class, turns out these are sort of misnamed as axioms, and
really are just definitions. Not on the same level as AoC, the bedrock
on which much of modern mathematics lies. 

\begin{theorem}
Let \(X\) be a topological space satisfying the \(T_1\) axiom. Let
\(A\subseteq X\). Then \(x\) is a limit point of \(A\) iff every
neighborhood of \(x\) contains infinitely many points of \(A\).
\end{theorem}

For a finite set in a \(T_1\) space, there cannot exist any limit
points, since \(A\) doesn't have infinitely many points. This leads
directly to \(A\) being closed, though, since it vacuously contains all
of its limits points (which don't exist).

\begin{proof}
First, if every neighborhood of \(x\) contains infinitely many points of
\(A\), then every neighborhood of \(x\) must contain at least one other
point of \(A\) than \(x\). \(x\) is then a limit point. 

Next, we consider \(x\) as a limit point of \(A\), with \(X\) as above. By
contradiction, assume that there exists a neighborhood \(U\), open in
\(X\), and \(B = U\cap A\setminus \{x\} = \{x_1, x_2,\ldots,x_n \}\) is
finite. Consider \(C = X\setminus B\), which is an open neighborhood of
\(x\). Now, \(D = U\cap C\) is also open, and contains \(x\). By
construction, this set \(D\) has empty intersection with \(A\setminus
\{x\}\). This is a contradiction, since \(x\) was assumed to be a limit
point. Thus, every open neighborhood of \(x\) intersects \(A\) at
infinitely many points.
\end{proof}

\begin{theorem}
Let \(X\) be a Hausdorff topological space. Then, any sequence
\(\{x_n\}_{n\in\mathbb{N}}\) converges to at most one point \(x \in X\).
\end{theorem}

\begin{proof}
Let \(\{x_n\}\) be a sequence of points in \(X\) which converges to
\(x\). Let \(x'\) be an element of \(X\), \(x' \not= x\). Let \(U\) and \(V\) be
disjoint open neighborhoods of \(x\) and \(x'\), respectively. Then, by
definition of convergence, there exists some large, finite integer N
such that \(x_n \in U\), \(\forall n > N\). Since \(U\) and \(V\) are
disjoint, these points cannot be in \(V\), and thus the sequence cannot
converge to \(x'\).
\end{proof}

\begin{definition}
In a Hausdorff space, the unique point to which a convergent sequence
converges is called \textbf{the limit}. 
\end{definition}

\begin{theorem}
Every simply ordered set is a Hausdorff space in the order
topology. The product of two Hausdorff spaces is Hausdorff. Subspaces
of Hausdorff spaces are Hausdorff. 
\end{theorem}

This particular theorem will come in handy for problems. Needs a user
supplied proof, though.

\section{Continuous Functions \textit{<2020-02-07 Fri 09:46>}}
\label{sec:org384eac9}
\begin{definition}
Let \(X,Y\) be topological spaces. A function \(f: X\rightarrow Y\) is
said to be \textbf{continuous} if for every open \(V \subseteq Y\) the
preimage \(f^{-1}(V)\) is open in \(X\). 
\end{definition}
\end{document}