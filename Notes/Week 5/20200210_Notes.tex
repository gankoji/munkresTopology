% Created 2020-02-14 Fri 09:48
% Intended LaTeX compiler: pdflatex
\documentclass[11pt]{article}
\usepackage[utf8]{inputenc}
\usepackage[T1]{fontenc}
\usepackage{graphicx}
\usepackage{grffile}
\usepackage{longtable}
\usepackage{wrapfig}
\usepackage{rotating}
\usepackage[normalem]{ulem}
\usepackage{amsmath}
\usepackage{textcomp}
\usepackage{amssymb}
\usepackage{capt-of}
\usepackage{hyperref}
\usepackage{amsthm}
\newtheorem{theorem}{Theorem}[section]
\newtheorem{lemma}{Lemma}[section]
\newtheorem{definition}{Definition}[section]
\newtheorem{question}{Question}[section]
\newtheorem{summary}{Summary}[section]
\newtheorem{corollary}{Corollary}[section]
\author{Jake Bailey}
\date{\today}
\title{Math 532 Notes}
\hypersetup{
 pdfauthor={Jake Bailey},
 pdftitle={Math 532 Notes},
 pdfkeywords={},
 pdfsubject={},
 pdfcreator={Emacs 26.3 (Org mode 9.3.3)}, 
 pdflang={English}}
\begin{document}

\maketitle
\tableofcontents


\section{Continuous Functions \textit{<2020-02-10 Mon 09:06>}}
\label{sec:orgaa8de10}
\begin{definition}
Let \(X,Y\) be topological spaces. A function \(f: X\rightarrow Y\) is
said to be \textbf{continuous} if for every open \(V \subseteq Y\) the
preimage \(f^{-1}(V)\) is open in \(X\). 
\end{definition}

Note: If topology is given by a basis, it suffices to show above
condition for the basis. Same goes for a subbasis. 

The above definition of continuity is equivalent to the \(\epsilon -
\delta\) definition of continuity in \(\mathbb{R}^n\).

\subsection{Examples}
\label{sec:org1cf1023}
The identity function \(id: \mathbb{R}\rightarrow\mathbb{R}_l\) is not
continuous, but \(id: \mathbb{R}_l \rightarrow \mathbb{R}\) is, due to
the inclusion of open sets of \(\mathbb{R}\) in \(\mathbb{R}_l\), but not
the other way around. 

\section{Exam Notes}
\label{sec:orgba069a6}
Expect 5-6 problems on day one, 1 extra for grad students on day
two. Still get a chance to rework one of the problems the second day.

A couple problems will be on definitions/theorems, and the rest on
problems like the homework.

Probably a good idea to work through some definition/theorem
flashcarding or the like this week, brush up.

\section{More on continuous functions}
\label{sec:orgb43d574}

\begin{theorem}
Let \(X\) and \(Y\) be topological spaces. The following are equivalent:
\begin{itemize}
\item \(f: X\rightarrow Y\) is continuous
\item \(\forall A \subseteq X\), \(f(\overline{A}) \subseteq \overline{f(A)}\)
\item \(\forall B \subseteq Y\), \(B\) closed, \(f^{-1}(B)\) is closed
\item \(\forall x \in X\), and for every neighborhood \(V\) of \(f(x)\),
\(\exists U\) (neighborhood of \(x\)) such that \(f(U) \subseteq V\).
\end{itemize}
\end{theorem}

For proof, we'll show 1 -> 2 -> 3 -> 1 = 4
\begin{proof}
1 implies 2: Assume \(f\) is continuous, and \(A\subseteq X\). Let
\(x\in\overline{A}\). Claim that \(f(x) \in f(\overline{A})\). Recall that
a point is in the closure of a set if every neighborhood of the point
intersects the set. Let \(V\) be a neighborhood of \(f(x)\). Then \(x \in
f^{-1}(V)\). But \(x \in \overline{A}\), so every neighborhood of \(x\)
intersects \(A\) at some point. Let \(y \in f^{-1}(V)\cap A\). Then, \(f(y)
\in V\cap f(A)\). So \(V\cap f(A) \not = \emptyset\), and thus \(f(x) \in
\overline{f(A)}\).

2 implies 3: Assume that \(f(\overline{A}) \subseteq
\overline{f(A)}\). Let \(B\subseteq Y\) be closed. Claim that \(A =
f^{-1}(B)\) is closed in \(X\), i.e. claim \(\overline{A} = A\). By set
theory we have \(f(A) = f(f^{-1}(B)) \subseteq B\). Let \(x \in
\overline{A}\). Then \(f(x) \in f(overline{A})\). By the above,
\(f(\overline{A})\subseteq \overline{f(A)}\subseteq \overline{B} =
B\). So \(x \in f^{1}(B) = A\). Thus, we've shown that \(\overline{A}
\subseteq A\), and we know that \(A \subseteq \overline{A}\), thus \(A =
\overline{A}\), and \(A\) is closed. 

3 implies 1: Assume that \(\forall B\subseteq Y\) closed, \(A =
f^{-1}(B)\) is also closed. Let \(U \subseteq Y\) be open. Then, claim
that \(f^{-1}(U)\) is open in \(X\). Let \(V = Y - U\) be the complement of
\(U\). \(V\) is closed, so by assumption, \(T = f^{-1}(V)\) is also
closed. But, \(T = f^{-1}(Y) - f^{-1}(U) = X - f^{-1}(U)\). Thus, \(C =
f^{-1}(U)\) is open, since its complement is closed. \(f\) is continuous. 

4 equals 1: Both ways

1 implies 4: Let \(f\) be continuous. Let \(x \in X, V_{f(x)}\subseteq
Y\). Claim that \(\exists U_x\) such that \(f(U_x) \subseteq V\). To see
this, take \(U_x = f^{-1}(V_{f(x)})\), which is an open neighborhood of
\(f(x)\). Then, \(f(U) \subseteq V\). 

4 implies 1: Assume that \(\forall x \in X\) and \(\forall V_{f(x)}
\subseteq Y\), \(\exists U_x\) such that \(f(U) \subseteq V\). Claim that
\(f\) is continuous, i.e. that \(f^{-1}(V)\) is open, for all open
\(V\). Let \(V \subseteq Y\) be open. Let \(f(x) \in V\). Let \(V_x\) be an
open neighborhood of \(f(x)\) such that \(V_x \subseteq V\). By our
assumption, \(\exists U_x \subseteq X\) open for each \(V_x\), and \(f(U_x)
\subseteq V_x\). Then, let \(U = \bigcup U_x\). This \(U\) is still open,
and it is the pre-image of \(V\) under \(f\). \(f\) is continuous. 
\end{proof}


\section{Homeomorphisms \textit{<2020-02-12 Wed 09:13>}}
\label{sec:org09f9925}

\begin{definition}
Let \(X\) and \(Y\) be topological spaces and \(f:X\rightarrow Y\) be
bijective (1-1 and onto). Then \(f\) is called a homeomorphism if and
only if both \(f\) and the inverse function \(f^{-1}:Y\rightarrow X\) are
continuous.
\end{definition}

NOTE: In homeworks and exam, don't need to prove that a function is
continuous. 

Equivalently, a bijection \(f:X\rightarrow Y\) is a homeomorphism if
\(f(U)\) is open iff \(U\) is open.

Note: A homeomorphism gives a bijection between both the spaces \(X\)
and \(Y\) and their topologies.

\begin{definition}
A property of a space expressed solely in terms of the topology
on the space is called a \textbf{topological property}.
\end{definition}

\begin{definition}
If \(f:X\rightarrow Y\) is an injective continuous map, and we consider
\(Z = f(X)\) as a subspace of \(Y\), then if \(f':X\rightarrow X\) is
defined as the restriction of \(f\)'s range to \(f(X) = Z\), then \(f'\) is
a homeomorphism, and we call \(f\) a \textbf{topological embedding}.
\end{definition}

\subsection{Examples}
\label{sec:org12fe265}

\(f:\mathbb{R}\rightarrow\mathbb{R}, f(x) = 3x + 1\) is a
homeomorphism. \(g(x) = \frac{1}{3}*(x - 1)\) is its inverse. 

A bijective function can be continuous and not be a
homeomorphism. Consider \(S^1\) (the unit circle) in the subspace
topology. Then, consider \(F: [0,1) \rightarrow S^1, F(t) = (cos 2\pi
t, sin 2 \pi t)\). Notice \([0, 1/4)\) is open in \([0,1)\), but
\(F([0,1/4))\) is not open in \(S^1\).

\(F:(-1, 1)\rightarrow \mathbb{R}, F(x) = \frac{x}{1 - x^2}\). Then,
\(G(y) = \frac{2y}{1 + (1 + 4y^2)^{1/2}}\). This bijection is order
preserving, so it's a homeomorphism. 

\subsection{Back to homeomorphism theorems}
\label{sec:orgcfb2a64}
\begin{theorem}
(Rules for constructing continuous functions): 

\begin{enumerate}
\item\relax [The constant function] If \(f:X\rightarrow Y\) maps all of \(X\) to a
single point \(y_0\in Y\), then \(f\) is continuous.
\item\relax [Inclusion] If \(A\) is a subspace of \(X\) the inclusion \(j:
   A\rightarrow X\) is continuous.
\item\relax [Composites] If \(f:X\rightarrow Y\), \(g:Y\rightarrow Z\) are
continuous, then \(g\circ f\) is continuous.
\item\relax [Restricting Domain] If \(f:X\rightarrow Y\) is continuous, and \(A\) a
subspace of \(X\), then \(f|A:A\rightarrow Y\) is continuous.
\item\relax [Restricting or Expanding the Codomain] of a continuous function
gives a continuous function.
\item\relax [Local Formulation of Continuity] The map \(f:X\rightarrow Y\) is
continuous if \(X\) can be written as a union of sets \(U_{\alpha}\)
such that \(f|U_{\alpha}\) is continuous for each \(\alpha\).
\end{enumerate}
\end{theorem}

\begin{proof}
(Just of part 6): Let \(U\subseteq Y\) be open. Then, \(f^{-1}(U) =
\bigcup f|U_{\alpha}^{-1}(U)\), and each \((f|U_{\alpha}(U))\) is open. 
\end{proof}

\begin{lemma}
(The Pasting Lemma) Let \(X = A\cup B\), where \(A\) and \(B\) are closed in
\(X\). Let \(f:A\rightarrow Y\), and \(g:B\rightarrow Y\) be continuous. If
\(f(x) = g(x)\) for all \(x \in A\cap B\), then \(h: X\rightarrow Y\), the
combination of the two, is continuous.
\end{lemma}

\begin{proof}
Let \(C \subseteq Y\) be closed. Then, \(h^{-1}(C) = f^{-1}(C)\cup
g^{-1}(C)\). \(f^{-1}(C)\) is closed in \(A\), same for \(g\) and \(B\). But,
\(A\) and \(B\) are both closed in \(X\), so the preimages of \(C\) are also
closed in \(X\) (what theorem is this?). So, \(f^{-1}(C)\cup g^{-1}(C)\)
is closed in \(X\), and \(h\) is continuous. 
\end{proof}

\section{More on Maps \textit{<2020-02-14 Fri 08:58>}}
\label{sec:org782f495}

\textbf{Three weeks until spring break. Starts 3/9/2020.}

\begin{theorem}
(Maps into Products) Let \(f: A\rightarrow X\times Y\) be given by \$
f(a) = (f\textsubscript{1}(a), f\textsubscript{2}(a))\$. Then \(f\) is continuous iff the coordinate
functions \(f_1:A\rightarrow X, f_2: A\rightarrow Y\) are continuous. 
\end{theorem}

\begin{proof}
First observe that \(\pi_1:X\times Y\rightarrow X\) and \(\pi_2:X\times
Y\rightarrow Y\) are continuous. E.g. \(U \subseteq X\) is open, then
\(\pi_1^{-1}(U) = U\times Y\) which is open. Note that for each \(a\in
A\), \(f_1(a) = \pi_1(f(a)), f_2(a) = \pi_2(f(a))\). So if \(f\) is
continuous, then \(f_1\) and \(f_2\) are compositions of continuous
functions, so they're continuous. 

Now, in the opposite direction, suppose that \(f_1\) and \(f_2\) are
continuous. Consider \(U\times V\) open in \(X\times Y\). The preimage is
\(f^{-1}(U\times V) = \{ a \in A\ |\ (f_1(a),f_2(a))\in U\times
V\}\). But \(f_1^{-1}(U) = \{ a \in A\ |\ f_1(a) \in U\}\), and
\(f_2^{-1}(U) = \{ a \in A\ |\ f_2(a) \in V \}\). Thus, \(f^{-1}(U\times
V) = \{ a \in A\ |\ (f_1(a) \in U) \text{and} (f_2(a) \in V) \} =
f_1^{-1}(U)\cap f_2^{-1}(V)\), which is the intersection of two open
sets (by assumption). Thus, the preimage under \(f\) is open, and
therefore \(f\) is continuous. 
\end{proof}

\section{The Product Topology in Detail \textit{<2020-02-14 Fri 09:11>}}
\label{sec:org3024cea}
Section 19 in Munkres.

Consider the Cartesian product \(X\times\ldots\times X_n\), \(n\) finite,
and \(X_1\times X_2\times\ldots\), where each \(X_i\) is a topological
space. Two possible topologies: 

\begin{enumerate}
\item Basis is all sets of the form \(U_1\times\ldots\times U_n\) in finite
case, or \(U_1\times U_2\times\ldots\) in infinite case where each
\(U_i\) is open in \(X_i\). (This is called a "Box Topology").
\item Take as subbasis all sets of the form \(\pi_i^{-1}(U)\) where \(U\) is
open in \(X_i\). This looks like the product of complete spaces with
\(U\) in place of the i-th component, i.e. \(X_1\times
   X_2\times\ldots U\times X_{i+1}\ldots\). This gives the product
topology.
\end{enumerate}

Note that in the case of a finite product, these two definitions are
equivalent. They only get weird and different when we go to infinite
products. A basis element in the product topology is a finite
intersection of subbasis elements generated as in 2). 

E.G \(\pi_{i_1}^{-1}(U_1)\cap \pi_{i_2}^{-1}(U_2)\cap\ldots\cap
\pi_{i_k}^{-1}(U_k) = X_1\times\ldots\times U_1\times
X_{i_1+1}\times\ldots\times U_2\times X_{i_2 + 1}\ldots\).

So \(\vec{x}\in B\) iff \(\pi_1(x)\in U_i\) for \(i = 1, \ldots k\). (So no
restriction on other components of \(\vec{x}\). Note: These
constructions give same topology for finite products, but differ for
infinite products. 

Notation: If \(J\) is an arbitrary index set, we say a function
\(x:J\rightarrow X\) is a J-tuple. We often write this function as
\((x_{\alpha})_{\alpha \in J}\). If \(\{A_{\alpha}\}_{\alpha \in J}\) is
an indexed family of sets, then the cartesian product of this indexed
family denoted as \(\Pi_{\alpha \in J}A_{\alpha}\) is the set of all
J-tuples \((x_{\alpha})_{\alpha \in J}\) of elements of \(X\) such that
\(x_{\alpha}\in A_{\alpha}\), \(\forall \alpha \in J\). (So its just \(x:
J\rightarrow \bigcup A_{\alpha}\) such that \(x(\alpha)\in A_{\alpha},
\forall\alpha)\).

\begin{definition}
\(\pi_{\beta}: \Pi\limits_{\alpha\in J}X_{\alpha}\rightarrow X_{\beta}\)
is called the projection mapping with index \(\beta\):
\(\pi_{\beta}(\vec{x}) = \vec{x}(\beta) = x_{\beta}\). 
\end{definition}

\begin{definition}
Let \(S_{\beta} = \{ \pi_{\beta}^{-1}(U_{\beta})\ |\ U_{\beta}
\text{open in} X_{\beta}\}\). Let \(SB = \bigcup\limits_{\beta \in
J}S_{\beta}\). The topology generated by this subbasis is called the
product topology. \(\Pi\limits_{\beta \in J} X_{\beta}\) with this
topology is called a product space.
\end{definition}

A typical basis element is of the form \(B =
\pi^{-1}_{\beta_1}(U_1)\cap\ldots\cap\pi^{-1}_{\beta_n}(U_n)\) where
\(\beta_1, \beta_2, \ldots \beta_n \in J\), and each \(U_i\) is open in
\(X_{\beta_i}\). We will often abbreviate this as \(B = \Pi\limits_{\beta
\in J}B_{\beta}\) where \(B_{\beta} = X_{\beta}\) if \(\beta\not =
\beta_1,\ldots\beta_n, B_{\beta} = U_i\) if \(\beta = \beta_i\).

Is the box topology finer than the product topology, or the other way
around? Being open in the product topology means being open in the box
topology, but not the other way around. This means the box topology is
generally finer than the product topology.

\begin{theorem}
Suppose the topology on each space \(X_{\alpha}\) is given by a basis
\(B_{\alpha}\). The collection of sets of the form \(|Pi\limits_{\alpha
\in J}b_{\alpha}\), where \(b_{\alpha}\in B_{\alpha}\) will serve as a
basis for the box topology. The collection of all sets of the same
form, where \(b_{\alpha}\in B_{\alpha}\) for finitely many \(\alpha\), and
\(B_{\alpha} = X_{\alpha}\) for the remaining indices, is a basis for
the product topology. 
\end{theorem}
\end{document}