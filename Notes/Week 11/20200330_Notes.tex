% Created 2020-04-03 Fri 09:45
% Intended LaTeX compiler: pdflatex
\documentclass[11pt]{article}
\usepackage[utf8]{inputenc}
\usepackage[T1]{fontenc}
\usepackage{graphicx}
\usepackage{grffile}
\usepackage{longtable}
\usepackage{wrapfig}
\usepackage{rotating}
\usepackage[normalem]{ulem}
\usepackage{amsmath}
\usepackage{textcomp}
\usepackage{amssymb}
\usepackage{capt-of}
\usepackage{hyperref}
\usepackage{amsthm}
\newtheorem{theorem}{Theorem}[section]
\newtheorem{lemma}{Lemma}[section]
\newtheorem{definition}{Definition}[section]
\newtheorem{question}{Question}[section]
\newtheorem{summary}{Summary}[section]
\newtheorem{corollary}{Corollary}[section]
\author{Jake Bailey}
\date{\today}
\title{Math 532 Notes}
\hypersetup{
 pdfauthor={Jake Bailey},
 pdftitle={Math 532 Notes},
 pdfkeywords={},
 pdfsubject={},
 pdfcreator={Emacs 26.3 (Org mode 9.3.6)}, 
 pdflang={English}}
\begin{document}

\maketitle
\tableofcontents

\section{Compact Subspaces of the Real Line \textit{<2020-03-30 Mon 09:03>}}
\label{sec:org79de15c}

\begin{theorem}
The extreme value theorem: Let \(f: X\rightarrow Y\) be continuous, where \(Y\) is
an ordered set in the order topology. If \(X\) is compact, then \(\exists c, d \in
X\) such that \(f(x) \leq f(x) \leq f(d)\). 
\end{theorem}

\begin{proof}
Assume \(f: X\rightarrow Y\) be continuous, \(X\) compact, \(Y\) ordered w/ the order
topology. Let \(A = f(X)\). The continuous image of a compact space is itself is
compact, so \(A\) is compact. Claim that \(A\) has a largest element \(M\) and a
smallest element \(m\). I.e. \(m, M \in A\), so \(\exists c, d \in X\) such that \(f(c)
= m, f(d) = M\).

If \(A\) has no largest element, then we can build an open cover which has no
finite subcover: \(\{(-\infty, a)\ |\ a\in A\}\) is an open cover of \(A\). But \(A\)
is compact, so it has a finite subcovering, say \(\{ (-\infty, a_1), \ldots,
(-\infty, a_n)\}\). Let \(a = \text{max}\{a_1, \ldots, a_n\}\). This \(a\in A\) is
not a member of any \((-\infyt, a_i)\). This is a contradiction, since these sets
were assumed to be a cover. So \(A\) has a largest element. The argument for a
smallest element is similar.  
\end{proof}

\begin{definition}
Let \((X, d)\) be a metric space. Let \(A\subseteq X, A\not = \emptyset\). For \(x\in
X\) define \(d(x,A) = \text{inf}\{d(x,y)\ |\ y\in A\}\) as the distance from \(x\) to
\(A\). 
\end{definition}

Note: for fixed \(A\) this is a continuous function of \(x\). Next, we want to find
a sort of diameter of containment of the set \(A\). We can do so by finding the
supremum of the distance between any two points in the set. 

\begin{definition}
Let \((X, d)\) be a metric space, \(A\subseteq X\), \(A\) bounded. The diameter of
\(A\) is: \(\text{diam} A = \text{sup}\{d(x,y)\ |\ x,y\in A\}\). 
\end{definition}

\begin{lemma}
The Lebesgue Number Lemma: Let \(A\) be an open covering of \((X,d)\), a metric
space. If \(X\) is compact, then \(\exists \delta > 0\), called a Lebesgue number,
such that every subset of \(X\) with diameter less than \(\delta\) is a subset of
some member of \(A\). 
\end{lemma}

Note: \(\delta\) is the Lebesgue number of \(A\). 

\begin{proof}
Let \(A\) be an open covering of \((X,d)\), a compact metric space. If \(X\in A\),
then any positive number will work as a Lebesgue number (and we're done). So,
now we assume \(X\not\in A\). 

Choose a finite subcollection of \(A\) which will still cover \(X\) (since \(X\) is
compact), say \(\{A_1,\ldots, A_n\}\). Each \(A_i\) is open, so its complement must
be closed. Let \(C_i = X\setminus A_i\) be these closed complements.

Define \(f:X\rightarrow\mathbb{R}\) as \(f(x) = 1/n\sum\limits_{i=1}^n d(x,C_i)\).
Essentially, we're taking the average distance of the closed sets to \(X\). First,
we claim that the average distance is not zero. Let \(x \in X\), and choose \(A_i\)
which contains \(x\). Choose \(\epsilon > 0\) so that \(B_d(x,\epsilon)\subseteq
A_i\). Thus, \(d(x, C_i)\geq \epsilon\). So \$f(x) \(\ge\) \(\epsilon\) /n. 

Now, we have that \(f\) is a continuous map from a compact set to an ordered one,
so by the extreme value theorem, it has both a minimum and a maximum. Call the
minimum \(\delta\), and claim that \(\delta\) is the Lebesgue number of \(A\). 

Let \(B\subseteq X\) with diam \(B < \delta\). Let \(x_o \in B\). Then, \(B\subseteq
B_d(x_0, \delta)\). Now, \(\delta \leq f(x_0) \leq d(x_0, C_m)\) where \(d(x_0,
C_m)\) is the largest of all the distances from \(x_0\) to each \(C_i\). 

So, \(B\subseteq B_d(X_0, \delta) \subseteq A_m = X\setminus C_m\). Thus, \(\delta\)
is the Lebesgue number of \(A\).  
\end{proof}

\begin{definition}
A function \(f:(X, d_x)\rightarrow (Y, d_y)\) between metric spaces is continuous
at \(x_0\in X\) if \(\forall \epsilon > 0, \exists \delta > 0\) such that \(\forall
y\in X, d_x(x_0, y) < \delta \Rightarrow d_y(f(x_0), f(y)) < \epsilon\). This is
a pointwise condition, where \(\delta\) depends on both \(\epsilon\) and \(x_0\).  
\end{definition}

\begin{definition}
A function \(f:(X, d_x)\rightarrow (Y, d_y)\) between metric spaces is said to be
uniformly continuous if \(\forall \epsilon > 0, \exists \delta > 0, \forall x_0,
x_1 \in X\) such that \(d_x(x_0, x_1)<\delta\Rightarrow d_y(f(x_0), f(x_1)) <
\epsilon\). 
\end{definition}

\begin{theorem}
The Uniform Continuity Theorem: If \(f:X\rightarrow Y\) is a continuous map from a
compact metric space to a metric space, then that map is uniformly continuous. 
\end{theorem}

\begin{proof}
Let \(\epsilon > 0\), cover \(y\) by sets \(B(y, \epsilon/2)\) for \(y\in Y\). Cover \(X\)
by \(A = \{ f^{-1}(B(y, \epsilon/2))\ |\ y\in Y\}\). Let \(\delta\) be the Lebesgue
number of \(A\). If \(x_1, x_2\in X\), and the \(d(x_1, x_2) < \delta\), then
diam\(\{x_1, x_2\} < \delta\), and this set is a subset of one of the covering
elements in \(A\), i.e. \(\{x_1, x_2\} \subseteq f^{-1}(B(y, \epsilon/2))\) for some
\(y\in Y\). Then, \(d_y(f(x_1), f(x_2))\leq d(f(x_1), y) + d(f(x_2), y) <
\epsilon/2 + \epsilon/2 = \epsilon\).  
\end{proof}
\section{Compact Subsets of the Real Line \textit{<2020-04-01 Wed 08:59>}}
\label{sec:orgb853557}

\begin{definition}
\(x\in X\) is isolated if \(\{x\}\) is open in \(X\). 
\end{definition}

\begin{definition}
A set \(A\) is countable if it is finite or countably infinite (i.e. there exists
a bijection from the set to the naturals).  
\end{definition}

\begin{theorem}
Let \(X\) be a nonempty compact Hausdorff space. If \(X\) has no isolated points,
then \(X\) is uncountable. 
\end{theorem}
\begin{proof}
Step 1: Show that given \(U\subseteq X\), open, nonempty, and taking \(x\in X\),
\(\exists V\subseteq U\), nonempty, open, such that \(x\not\in \overline{V}\). Let
\(y \in U\), \(y \not\in x\). If \(x \in U\) since \(x\) is not isolated, \(U \not
\{x\}\), so \(\exists y\in U, y\not = x\). If \(x \not\nit U\), \(y\) exists since
\(U\not = \emptyset\). 

Choose \(W_1, W_2\) open such that \(x\in W_1\), \(y\in W_2\). Then \(V = W_2\cap U\) is
the desired open set.  

Step 2: Show that a function \(f: \mathbb{Z}_+ \rightarrow X\) cannot be
surjective. Then it follows that \(X\) is uncountable. Let \(x_n = f(n)\). Apply
step 1 with \(x_1 = x, U = X_1\) to obtain nonempty open \(V_1\) such that \(x_1
\not\in \overline{V}_1\). We are going to apply this recursively, choosing
\(U_{n+1} = V_n\). In general, \(V_n \subseteq V_{n-1}, x_n \not\in
\overline{V}_n\). We have \(\overline{V}_1 \superset \overline{V}_2 \ldots\), etc.
Thus we have a descending sequence of nonempty closed sets.

Because \(X\) is compact, \(\{\overline{V}_n\ |\ n\in\mathbb{N}\}\) is a collection
of closed sets with the Finite Intersection Property, and \(\exists
x\in\cap\limits_n \overline{V}_n\). Then, \(\forall n, x\not = x_n\) since
\(x\in\overline{V}_n\) and \(x_n\not\in\overline{V}_n\).  
\end{proof}

One corollary of this: Every closed interval in \(\mathbb{R}\) is uncountable. 
\section{Limit Point Compactness \textit{<2020-04-01 Wed 09:37>}}
\label{sec:org463da95}
\begin{definition}
A space \(X\) is said to be limit point compact if every infinite subset of \(X\)
has a limit point. 
\end{definition}

\begin{theorem}
Compactness implies limit point compactness. 
\end{theorem}

\begin{proof}
Let \(X\) be compact. Suppose \(A\subseteq X\) has no limit point (i.e. \(A\) is
closed, since it contains all (zero) of its limit points). For each \(a\in A\),
choose open \(U_a\) such that \(a\in U_a\), and \((U_a\setminus \{a\})\cap A =
\emptyset\). Notice \(\{X\setminus A\}\cup\{U_a\ |\ a\in A\}\) is an open cover of
\(X\), so it has a finite subcover (because \(X\) is compact). Notice \(X\setminus A\)
does not intersect \(A\), each \(U_a\) contains only one point of \(A\), and there can
only be finitely many of them (finite subcover), so \(A\) must be finite. Thus,
for a subset of \(X\) to not have a limit point, it must be finite, and all
infinite subsets of \(X\) must have a limit point.      
\end{proof}

Note that the converse (limit point compactness implies compactness) is not
true! Example: Let \(Y\) be a two point set in the indiscrete topology. Consider
\(X = \mathbb{Z}_+\times Y\). Every nonempty subset of \(X\) has a limit point. But,
\(U_n = \{n\}\times Y\) is an open cover with a finite subcover. 
\section{More Limit Point Compactness \textit{<2020-04-03 Fri 09:07>}}
\label{sec:orgf5b802a}
\begin{definition}
Let \(X\) be a well-ordered. Given \(\alpha\in X\), let \(S_{\alpha}\) denote
\(S_{\alpha} = \{ x\ |\ x\in X, x < \alpha\}\). This is the section of \(X\) by
\(\alpha\).
\end{definition}

\begin{lemma}
There exists a well-ordered set \(A\) having largest element \(\Omega\) so each
section \(S_{\Omega}\) of \(A\) by \(\Omega\) is uncountable, but evey other section of
\(A\) is countable. 
\end{lemma}

\begin{proof}
Begin with an uncountable well ordered set \(B\). Let \(C = \{1,2\}\times B\) in the
dictionary order topology. Some section of \(C\) has to be uncountable. any
\(1\times b\) for \(b\in B\) is less than any \(2\times b\). A section by \(\alpha =
2\times b\) is uncountable. 
Let \(\Omega\) be the least element of \(X\) such that the section of \(C\) by \(\Omega\) is
uncountable. \(A\) is the section of \(C\) together with \(\Omega\).  
\end{proof}

\(S_{\Omega}\) is uncountable, well ordered, and every other section is countable,
so it is called the minimum uncountable well ordered set. Denote \(A =
S_{\Omega}\cup\{\Omega\}\) by \(\overline{S}_{\Omega}\). 

\begin{theorem}
If \(A\) is a countable subset of \(S_{\Omega}\), then \(A\) has an upper bound in
\(S_{\Omega}\). 
\end{theorem}

Is \(S_{\Omega}\) compact? Take \(S_{\alpha}\) for \(\alpha\in A\). \(S_{\Omega}\) has
no largest element, so no finite subcover of this cover. So \(S_{\Omega}\) is not
compact. \(S_{\Omega}\) is limit point compact: Let \(A\subseteq S_{\Omega}\) be
infinite. Let \(B\subseteq A\) be countably infinite. Let \(b\) be an upper bound of
\(B\). So \(B\subseteq [a_o, b]\) where \(a_0 = \text{min} S_{\Omega}\). Since
\(S_{\Omega}\) has a least upper bound, \([a_0, b]\) is compact. So \(B\) has a limit
point in \([a_0, b]\). Then \(x\) is also a limit point of \(A\). So \(S_{\Omega}\) is
limit point compact.

\begin{definition}
Let \((x_n)_{n\in \mathbb{Z}_+}\) be a sequence of elements of the topological
space \(X\). Let \$n\textsubscript{1}<n\textsubscript{2}<\ldots < n\textsubscript{i} < \ldots \$ be an increasing sequence of
positive integers. Then \((y_i)_{n\in\mathbb{Z}_+}\) defined as \(y_i = x_{n_i}\)
is a subsequence of \((x_n)\).  
\end{definition}

\(X\) is sequentially compact if every sequence of points of \(X\) has a convergent
subsequence. 

\begin{theorem}
Let \(X\) be a metrizable space. The following are equivalent:

\begin{enumerate}
\item \(X\) is compact.
\item \(X\) is limit point compact.
\item \(X\) is sequentially compact.
\end{enumerate}
\end{theorem}

\begin{proof}
\begin{enumerate}
\item implies 2): done above!

\item implies 3): Assume \(X\) is limit point compact. Let \((x_n)_{n\in\mathbb{Z}_+}\)
be a sequence of elements of \(X\).
\end{enumerate}
\end{proof}
\end{document}
