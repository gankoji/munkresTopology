% Created 2020-02-18 Tue 14:31
% Intended LaTeX compiler: pdflatex
\documentclass[11pt]{article}
\usepackage[utf8]{inputenc}
\usepackage[T1]{fontenc}
\usepackage{graphicx}
\usepackage{grffile}
\usepackage{longtable}
\usepackage{wrapfig}
\usepackage{rotating}
\usepackage[normalem]{ulem}
\usepackage{amsmath}
\usepackage{textcomp}
\usepackage{amssymb}
\usepackage{capt-of}
\usepackage{hyperref}
\usepackage{amsthm}
\usepackage{amsmath}
\usepackage{amssymb}
\usepackage{graphicx}
\usepackage{fancyhdr}
\pagestyle{fancy}
\fancyhf{}
\rhead{Homework 4, Math 532}
\lhead{Jake Bailey}
\rfoot{Page \thepage}
\newtheorem{definition}{Definition}[section]
\author{Jacob Bailey}
\date{\today}
\title{Homework Set 4}
\hypersetup{
 pdfauthor={Jacob Bailey},
 pdftitle={Homework Set 4},
 pdfkeywords={},
 pdfsubject={},
 pdfcreator={Emacs 28.0.50 (Org mode 9.3.1)}, 
 pdflang={English}}
\begin{document}

\maketitle

\section{Problem 1}
\label{sec:orge0a8a35}

Show that \(X\) is Hausdorff if and only if the digaonal \(\Delta = \{\
x\times x\ |\ x\in X\}\) is closed in \(X\times X\). 

This problem was solved in Homework 3. Per Dr. Aubrey's instructions,
we're skipping it this time around. 

\section{Problem 2}
\label{sec:orge9794e6}

Show that the \(T_1\) axiom is equivalent to the condition that for each
pair of points in \(X\), each has a neighborhood not containing the
other. 

\begin{proof}
First, let the \(T_1\) axiom hold, such that for our given topological
space \(X\), finite point sets are closed. Then, let \(A = \{ x_1,
x_2\}\). \(A\) is closed by \(T_1\), and thus \(U = X\setminus A\) is
open. We then consider the set \(B = U\cup \{x_1\}\). \(B\) is also open,
since its complement, \(X\setminus B = \{x_2\}\) is finite, and
therefore closed. We can similarly construct a set \(C = U\cup
\{x_2\}\), which is again open. Thus, we have constructed neighborhoods
of \(x_1, x_2\) which do not contain the other point. 

Conversely, let \(\exists U,V\subseteq X\), \(x_1\in U\), \(x_2\in V\),
\(x_2\not\in U\), \(x_1\not\in V\), and \(U,V\) open. We can combine sets as
\(C = ((-\infty, x_2)\cup U \cup (x_2, \infty))\cap X\), and \(D =
((-\infty, x_1)\cup V \cup (x_1, \infty))\cap X\). Both of these are
open, since unions of open sets are open. Their complements, \(C^c,
D^c\), are thus closed. They are also finite (single) point sets,
effectively \(C^c = \{x_2\}\) and \(D^c = \{x_1\}\). Since \(X\) is a
topological space, finitely many unions of similarly constructed
closed sets will also be closed. We can therefore construct any set of
finitely many points which will be closed. Thus, \(T_1\) holds.
\end{proof}
\section{Problem 3}
\label{sec:orgffc9b19}

If \(A\subseteq X\), we define the \textbf{boundary} of \(A\) by the
equation

\(\text{Bd}\ A = \overline{A}\cap\overline{(X\setminus A)}\).

a) Show that Int \(A\) and Bd \(A\) are disjoint and that \(\overline{A} =
   \text{Int} A\cup\text{Bd}A\).

\begin{proof}


First, we note that Int \(A\) is the union of all open sets contained
within \(A\), and that Int \(A \subseteq \overline{A}\). If \(A\) is open,
then Int \(A = A\), \(\overline{(X\setminus A)} = (X\setminus A)\), and
\((X\setminus A)\cap A = \emptyset\). If \(A\) is closed, then \(A =
\overline{A}\), and \((X\setminus A)\) is open. Further, since Int \(A
\subset A \subseteq \overline{A}\), \(\text{Int}\ A\cap \text{Bd}\ A =
\emptyset\). Thus, Int \(A\) and Bd \(A\) are disjoint. 
\end{proof}

b) Show that Bd \(A = \emptyset\) iff \(A\) is both open and closed.

\begin{proof}
If \(A\) is closed, then \(A = \overline{A}\). If \(A\) is open, then
\(X\setminus A\) is closed, and also equal to its closure. Then, Bd \(A\)
= \(\overline{A}\cap\overline{(X\setminus A)}\) = \(A\cap(X\setminus A)\)
= \(\emptyset\). 

If Bd \(A\) = \(\emptyset\), then \(\overline{A}\cap\overline{X\setminus A}
= \emptyset\). This implies \(A = \overline{A}\), and \((X\setminus A) =
\overline{(X\setminus A)}\). Thus, \(A\) is both closed and open. 
\end{proof}

c) Show that \(U\) is open iff Bd \(U = \overline{U}\setminus U\).

\begin{proof}
If \(U\) is open, \(X\setminus U\) is closed. Bd \(U\) =
\(\overline{U}\cap\overline{(X\setminus U)} =
\overline{U}\cap(X\setminus U) = (\overline{U}\cap X)\setminus
(\overline{U}\cap U) = \overline{U}\setminus((U\cup U')\cap U) =
\overline{U}\setminus U\).

Next, let Bd \(U\) = \(\overline{U}\setminus U\). But, Bd \(U\) =
\(\overline{U}\cap\overline{(X\setminus U)}\). We can rewrite the first
equation as Bd \(U\) = \((\overline{U}\cap X)\setminus (\overline{U}\cap
U)\), which is equivalent to \(\overline{U}\cap (X\setminus U)\), as
shown in the preceding paragraph. Then, we have that
\(\overline{U}\cap\overline{(X\setminus U)} = \overline{U}\cap
(X\setminus U)\), such that \(\overline{(X\setminus U)} = (X\setminus
U)\), and thus \((X\setminus U)\) is closed. \(U\) is open. 
\end{proof}

d) If \(U\) is open, is it true that \(U = \text{Int}\ \overline{U}\)? Justify
   your answer. 

By definition, Int \(U\) is the union of all open sets contained in
\(U\). If \(U\) is open, then Int \(U\) = \(U\). We know from theorem 17.6 of
Munkres that \(\overline{U} = U\cup U'\), where \(U'\) is the set of all
limit points of \(U\). We also know that adding the limit points to the
set creates a closed set, not an open one, so we conclude that \(U\) is
the "biggest" open set contained within \(\overline{U}\). 

\section{Problem 4}
\label{sec:orgbf253fb}

Prove that for functions \(F:\mathbb{R}\rightarrow\mathbb{R}\) the
\(\epsilon - \delta\) definition of continuity implies the open set
definition. 

\begin{proof}
Assume \(f\) is continuous. Thus, by the "\(\delta - \epsilon\)"
definition of continuity, we have that \(\forall \epsilon > 0, \exists
\delta > 0\) such that if \(|x - y| < \delta\), then \(|f(x) - f(y)| <
\epsilon\).

Let \(U\subseteq\mathbb{R}\) be open. Then, for any \(y\in U\), \(\exists
x\) such that \(f(x) = y\), and there are neighborhoods \(y \in U_y\) and
\(x\in V_x\). Additionally, we can choose to let \(U = \bigcup U_y\) for
our chosen \(y\), and since the \(U_y\) are open (in the standard
topology, assumed here), \(U\) is also open. Finally, we note that all
of the \(V_x\) are open in the standard topology, so their union \(V =
\bigcup V_x\) is as well. But, by construction, \(V = f^{-1}(U)\), so the
preimage of an open set is open. 
\end{proof}

\section{Problem 5}
\label{sec:orgfada998}

Let \(a,b\), and \(c\) be real numbers with \(a\leq b\leq c\), and \(a <
c\). Let \(X\) denote the set \([a,c]\cup\{b'\}\), where \$[a,c] denotes a
closed interval in the real line and \(b'\) is a point not in
\([a,c]\). Let \(F\) be the family of subsets of \(X\) consisting of all
open subsets of \([a,c]\) together with all subsets of the form
\((U\setminus \{b\})\cup\{b'\}\), where \(U\) is an open subset of \([a,c]\)
\textbf{which contains $b$}. (Emphasis mine). 

a) Show that \(F\) is a basis for a topology on \(X\).
\begin{proof}
For notation's sake, we split \(F\) into two categories of subsets: \(U\)
the open subsets of the interval \([a,c]\), and \(W\), the collection of
sets of the form \((U\setminus \{b\})\cup\{b'\}\). 

We consider the two pieces of the definition of basis
separately. Trivially, we note that elements of \(F\) cover \(X\), for if
\(x \in [a,c]\), \(\exists U\in F\) such that \(x \in U\), and if \(x = b'\),
\(\exists W\in F\) such that \(x \in W\).

Next, we look to intersections of elements of \(F\). Let \(x \in X, x \in
f_1\cap f_2\). Either \(x \in [a,c]\), or \(x = b'\). If \(x \in [a,c]\),
\(f_1,f_2 \subseteq [a,c]\). Thus, \(\exists f_3 \subseteq [a,c]\) such
that \(x \in f_3 \subseteq f_1\cap f_2\). 

Finally, consider when \(x = b'\). Then, \(f_1, f_2 \in W\). We see that
\(\exists f_3 \in W\) such that \(b'\in f_3 \subseteq f_1\cap f_2\), and
we're done.
\end{proof}

b) Show that the map which interchanges \(b\) and \(b'\) and is the
   identity elsewhere is a homeomorphism.

\begin{proof}
Clearly, the map \(f:X\rightarrow X\) described above is a
bijection. Further, we also see that \(f = f^{-1}\). It is then enough
to show that \(f\) is continuous. Let \(U\subseteq X\) be open. Then, we
have four cases: \(\{b, b'\}\cap U = \{\emptyset, \{b\}, \{b'\},
\{b,b'\}\}\).

Case 1: \(f^{-1}(U) = U\). Done. 

Case 2: \(f^{-1}(U) = (U\setminus\{b\})\cup\{b'\} \in W\), which is open. 

Case 3: \(f^{-1}(U) = (U\setminus\{b'\})\cup\{b\} \subseteq [a,c]\),
which is also open. 

Case 4: \(f^{-1}(U) = U\). Done. 
\end{proof}

c) Show that this topology on \(X\) is not Hausdorff

\begin{proof}


Consider the two points \(b\) and \(b'\). By construction of \(X\), any
neighborhood of \(b'\) must also contain some neighborhood of \(b\), no
matter how small the neighborhood (see my emphasis in the question
text above). \(X\) is not Hausdorff. 
\end{proof}

d) Show that if \(f: X\rightarrow \mathbb{R}\) is continuous, then \(f(b)
   = f(b')\).

\begin{proof}
We'll assume \(\mathbb{R}\) to have the standard topology here. Using
the topological definition of continuity, we have that \(\forall
U\subseteq\mathbb{R}\), \(U\) open, \(V = f^{-1}(U)\) is also open.

Any open set containing \(b'\) also includes \((b - \epsilon,
b)\cup(b, b+\epsilon)\), by construction of \(X\). Assume \(f(b) \not =
f(b')\). Then \(\exists U, U'\), \(f(b)\in U, f(b')\in U'\), and \(U\cap U'
= \emptyset\), since \(\mathbb{R}\) is Hausdorff. But, \(f^{-1}(U)\cap
f^{-1}(U') \not = \emptyset\), so \(\exists x\) such that \(f(x) \in U\)
and \(f(x) \in U'\). Since \(U\cap U' = \emptyset\), this implies that \(x\)
maps to two distinct points in \(\mathbb{R}\), a violation of the
function rule. This is a contradiction, so \(f(b) = f(b')\). 
\end{proof}
\end{document}