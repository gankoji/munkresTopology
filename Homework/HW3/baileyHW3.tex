% Created 2020-02-08 Sat 17:10
% Intended LaTeX compiler: pdflatex
\documentclass[11pt]{article}
\usepackage[utf8]{inputenc}
\usepackage[T1]{fontenc}
\usepackage{graphicx}
\usepackage{grffile}
\usepackage{longtable}
\usepackage{wrapfig}
\usepackage{rotating}
\usepackage[normalem]{ulem}
\usepackage{amsmath}
\usepackage{textcomp}
\usepackage{amssymb}
\usepackage{capt-of}
\usepackage{hyperref}
\usepackage{amsthm}
\usepackage{amsmath}
\usepackage{amssymb}
\usepackage{graphicx}
\usepackage{fancyhdr}
\pagestyle{fancy}
\fancyhf{}
\rhead{Homework 3, Math 532}
\lhead{Jake Bailey}
\rfoot{Page \thepage}
\newtheorem{definition}{Definition}[section]
\author{Jacob Bailey}
\date{\today}
\title{Homework Set 3}
\hypersetup{
 pdfauthor={Jacob Bailey},
 pdftitle={Homework Set 3},
 pdfkeywords={},
 pdfsubject={},
 pdfcreator={Emacs 28.0.50 (Org mode 9.3.1)}, 
 pdflang={English}}
\begin{document}

\maketitle

\section{Problem 1}
\label{sec:org7fad07b}

Let \(A,B\), and \(A_{\alpha}\) denote subsets of a topological space
\(X\). Determine which of the following equations hold; if an equality
fails, give an example, and determine whether one of the inclusions
\(\supset\) or \(\subset\) holds.

\begin{enumerate}
\item \(\overline{A\cap B} = \overline{A}\cap\overline{B}\)
\item \(\overline{\cap A_{\alpha}} = \cap\overline{A_{\alpha}}\)
\item \(\overline{A\setminus B} = \overline{A}\setminus\overline{B}\)
\end{enumerate}


\subsection{\(\overline{A\cap B} = \overline{A}\cap\overline{B}\)}
\label{sec:org90fa121}
This equality holds. The intersection of two closed sets equals a
closed set, since the complement of an intersection is the union of
the complements, and the union of open sets is open in a topological
space. Since the two sides have the same interiors and are both
closed, they are equal. 

\subsection{\(\overline{\cap A_{\alpha}} = \cap\overline{A_{\alpha}}\)}
\label{sec:org84de64e}
This equality also holds. Argument is the same as above, since
topologies are closed under arbitrary unions of open sets (thus the
arbitrary intersection of closed sets is closed). 

\subsection{\(\overline{A\setminus B} = \overline{A}\setminus\overline{B}\)}
\label{sec:orgab81722}
This does not hold. \(\overline{A}\setminus\overline{B} \subset
\overline{A\setminus B}\). Consider, as example, \(A = (0,2)\) and \(B =
(1,2)\). Then \(\overline{A\setminus B} = [0,1]\), but
\(\overline{A}\setminus\overline{B} = [0,1)\).

\section{Problem 2}
\label{sec:org3c1e84f}
Show that every order topology is Hausdorff. 

\begin{definition}
A topological space is said to be \textbf{Hausdorff} if, for any two
distinct points \(x_1, x_2 \in X\), there exist open neighborhoods \(U_1,
U_2\) of\(x_1, x_2\) respectively, such that \(U_1\cap U_2 = \emptyset\).
\end{definition}

\begin{proof}
Let \(X\) be a topological space with the order topology, and let \(x,y
\in X\), distinct, such that \$ x < y \$. Then, we have two cases:

Case 1: \(\exists z\) such that \(x < z < y\). Then, \(U = (-\infty, z)\) is
open and contains \(x\). Similarly, \(V = (z, \infty)\) is open and
contains \(y\). But, by construction, \(U\cap V = \emptyset\). 

Case 2: Such a \(z\) does not exist (i.e. \(y\) is the "next point" in the
order after \(x\)). Then, We modify \(U\) and \(V\) as \(U = (-\infty, y) =
(-\infty, x]\), and \(V = (x, \infty) = [y, \infty)\). Clearly, we still
have that \(U\cap V = \emptyset\). 

Since our \(x,y\) are assumed distinct and arbitrary, this shows that
\(X\) must be Hausdorff. 
\end{proof}

\section{Problem 3}
\label{sec:org3acfebd}
Show that \(X\) is Hausdorff iff the diagonal \(\Delta = \{ x\times x\ |\
x \in X\}\) is closed in \(X\times X\).

\begin{proof}
Let \(X\times X\) have the product topology, and let \(\Delta\) be closed
in \(X\times X\). Choose \((x,y)\) in \(X\times X\), such that \(x \not =
y\). \((x,y) \in X\times X\setminus\Delta\). Then, there exist open
neighborhoods \(U\) and \(V\) of \(x\) and \(y\), respectively, such that
\((U\times V)\cap \Delta = \emptyset\), since \(X\times X\setminus\Delta\)
is open, due to \(\Delta\)'s closedness. Consequently, \(U\cap V =
\emptyset\), for if there was an \(x\) such that \(x \in U\cap V\), \((x,x)
\in (U\times V) \subseteq \Delta\), a contradiction. \(X\) is Hausdorff.

Next, let \(X\) be Hausdorff. Take \(x\) and \(y\) as above, such that
\((x,y) \in X\times X\setminus\Delta\). Since \(X\) is Hausdorff, we have
that \(\exists U,V \ni U\cap V = \emptyset\), \(U,V\) open. Since the
product topology is closed under unions, we can build up all of
\(X\times X\setminus\Delta\) via these neighborhoods, for all distinct
\(x,y \in X\). Thus, \(X\times X\setminus\Delta\) is open, and its
complement, \(\Delta\), is therefore closed. 
\end{proof}

\section{Problem 4}
\label{sec:org1a1cbe3}
In the finite complement topology on \(\mathbb{R}\), to what point or
points does the sequence \(x_n = 1/n\) converge? 

The sequence converges (in the topological sense, i.e. for every open
neighborhood \(U\) of \(x\), \(\exists N \in \mathbb{N}\) such that \(\forall
n > N, x_n \in U\)) to every point of \(\mathbb{R}\). This is because the
open neighborhoods must all be infinite (and, in fact, of infinite
measure), since they contain all but finitely many points of
\(\mathbb{R}\). Thus, for any point in \(\mathbb{R}\), we can find a
suitable \(N\) such that \(x_N\) is in its neighborhood.

\section{Problem 5}
\label{sec:org799cb9b}
Consider the lower limit topology on \(\mathbb{R}\) and the topology
given by the basis of Exercise 8 in Section 13. Determine the closures
of the intervals \(A = (0, \sqrt{2})\) and \(B = (\sqrt{2}, 3)\) in these
two topologies.

In the lower limit topology, \(\overline{A} = [0, \sqrt{2})\) and
\(\overline{B} = [\sqrt{2}, 3)\). 

In the topology defined in exercise 8 of section 13, both are the
same.
\end{document}