% Created 2020-01-23 Thu 20:45
% Intended LaTeX compiler: pdflatex
\documentclass[11pt]{article}
\usepackage[utf8]{inputenc}
\usepackage[T1]{fontenc}
\usepackage{graphicx}
\usepackage{grffile}
\usepackage{longtable}
\usepackage{wrapfig}
\usepackage{rotating}
\usepackage[normalem]{ulem}
\usepackage{amsmath}
\usepackage{textcomp}
\usepackage{amssymb}
\usepackage{capt-of}
\usepackage{hyperref}
\usepackage{amsthm}
\usepackage{amsmath}
\usepackage{amssymb}
\usepackage{graphicx}
\usepackage{fancyhdr}
\pagestyle{fancy}
\fancyhf{}
\rhead{Homework 1, Math 532}
\lhead{Jake Bailey}
\rfoot{Page \thepage}
\author{Jacob Bailey}
\date{\today}
\title{Homework Set 1}
\hypersetup{
 pdfauthor={Jacob Bailey},
 pdftitle={Homework Set 1},
 pdfkeywords={},
 pdfsubject={},
 pdfcreator={Emacs 26.3 (Org mode 9.3.1)}, 
 pdflang={English}}
\begin{document}

\maketitle

\section{Problem 1}
\label{sec:org49256bd}

Let \(T_1\) and \(T_2\) be topologies on a set \(X\). Show that \(T_1 \cap
T_2\) is also a topology on \(X\). Give an example to show that \(T_1\cup
T_2\) need not be a topology on \(X\). 

Recall that by definition, \(T_1\) and \(T_2\) both, as topologies,
satisfy the following properties: 

\begin{itemize}
\item Both include \(\emptyset\) and \(X\)
\item Both are closed under arbitrary unions
\item Both are closed under finite intersections
\end{itemize}

\begin{proof}
Consider \(T_3 = T_1 \cap T_2\). It is immediate that \(T_3\) contains
both \(\emptyset\) and \(X\), since \(T_1\) and \(T_2\) do. 

Next, we consider unions of members of \(T_3\). By definition, for a set
\(U_{\alpha} \in T_3\), \(U_{\alpha}\) must also be in \(T_1\) and
\(T_2\). Since both constituents are closed under unions, we have that
for any \(U_{3} = \bigcup\limits_{\alpha \in J} U_{\alpha}\), \(U_3\) is
also in \(T_3\) by closedness of \(T_1\) and \(T_2\). Hence, \(T_3\) is closed
under arbitrary unions.

Finally, we consider finite intersections. Again, by definition of
\(T_3\), we have that for any set \(U \in T_3\), \((U \in T_1)\ \text{and}\
(U \in T_2)\). Thus, for a set \(V = \bigcap\limits_{\alpha \in
J}U_{\alpha}\), \(J\) finite, \((V \in T_1)\ \text{and}\ (V \in
T_2)\). Hence, \(V \in T_3\), and \(T_3\) is closed under finite
intersections.

We have shown that \(T_3\) is a collection of sets on \(X\) which
satisfies the three properties of a topology. Hence, \(T_3\) is a
topology on \(X\). 
\end{proof}

For an example of how \(T_1\cup T_2\) need not be a topology, consider
\(\mathbb{R}_{std} \cup \mathbb{R}_l\). We choose the sets \((1,2)\) and
\([3,4)\), which are in \(\mathbb{R}_{std}\) and \(\mathbb{R}_l\),
respectively. However, their union \((1,2)\cup [3,4)\) is not in \(T_3\),
since it is the union of the collections of sets, not the sets
themselves. Thus, \(T_3\) isn't closed under unions, and cannot be a
topology.

\section{Problem 2}
\label{sec:org509ef46}

Show that the topologies of \(\mathbb{R}_l\) and \(\mathbb{R}_K\) are not
comparable.

\begin{proof}
First, we recall the definition of \(\mathbb{R}_K\): the topology
generated by the basis of sets \(B = \{ (a,b) - K\ |\ a,b \in
\mathbb{R}\}\), where \(K = \{ \frac{1}{n} |\ n \in \mathbb{N}\}\).

Given an element of the basis of \(\mathbb{R}_l\), say \(b_l =
[a,b)\). Then, choose \(x = a\). Clearly, there is no open set in the
topology generated by \(B_K\) that both contains \(x\) and is a subset of
\([a,b)\).

Similarly, take \(b_K = (-1, 1) - K\), and let \(x = 0\). Then, there is
no element of \(B_l\) that both contains \(x\) and lies within \(b_K\),
since there is a point \(y = 1/n, n \in \mathbb{N}\) that lies
arbitrarily close to \(x\) which is not in \(b_K\). 

Thus, \(\mathbb{R}_l\) and \(\mathbb{R}_K\) are not comparable.
\end{proof}

\section{Problem 3}
\label{sec:org285ebe2}

\subsection{Part a}
\label{sec:org20dbca1}
Apply lemma 13.2 to show that the countable collection \(B = \{
(a,b) |\ a<b,\ a,b \in \mathbb{Q}\}\) is a basis that generates
\(\mathbb{R}_{std}\).

Lemma 13.2: Let \(X\) be a topological space. Suppose that \(C\) is a
collection of open sets, such that \(\forall U \in X\) and \(\forall x
\in U\), \(\exists c \in C\) such that \(x \in C \subseteq U\). Then \(C\) is
a basis for the topology in \(X\).

\begin{proof}


Define \(B_{\mathbb{Q}} = \{ (a,b) |\ a < b, a,b \in \mathbb{Q}\}\).

Let \(U\) be in the topology generated by \(B_{\mathbb{Q}}\). Then, \(U\) is
some combination of finite unions and arbitrary intersections of
elements of \(B_{\mathbb{Q}}\), which are all also open sets in
\(\mathbb{R}_{std}\). Thus, \(U \in \mathbb{R}_{std}\). 

Conversely, let \(U\) be in \(T_{std}\), and take \(x \in U\). Then,
\(\exists (a,b) \in \mathbb{R}\) such that \(x \in (a,b) \subseteq
U\). Since \(\mathbb{Q}\) is dense in \(\mathbb{R}\), we can find \(c,d\)
such that \(a < c < x < d < b\), and thus \(x \in (c,d) \subseteq
U\). Thus, by lemma 13.2, our collection \(B_{\mathbb{Q}}\) is a basis
for the standard topology in \(\mathbb{R}\).
\end{proof}

\subsection{Part b}
\label{sec:org067b9e7}
Show that the collection \(C = \{[a,b)|\ a<b,\ a,b \in \mathbb{Q}\}\) is a
basis that generates a topology different from \(\mathbb{R}_l\).

\begin{proof}


Clearly, \(C\) is a basis for a topology in \(\mathbb{R}\), again by lemma
13.2. However, it cannot generate \([i,j), i,j
\not\in\mathbb{Q}\). \([i,j) \not\in C\), since \(C\) contains only
intervals with rational endpoints. Further, we cannot construct
\([i,j)\) as \([a,b)\cap[c,d)\), since \([a,b)\cap[c,d) = [\text{max}(a,c),
\text{min}(b,d))\), and \(a,b,c,d \in \mathbb{Q}\). Finally, we note that
we also cannot construct \([i,j)\) as a union of intervals \([a_i, b_i),
a_i, b_i \in\mathbb{Q}\), for the same reason as stated above. Since a
topology must be closed under arbitrary unions and finite
intersections, and we cannot construct \([i,j), i,j \not\in\mathbb{Q}\)
from \(C\), the topology generated by \(C\) cannot be the lower limit
topology on \(\mathbb{R}\).
\end{proof}

\section{Problem 4}
\label{sec:orgaa883d4}

Show that if \(A\) is a basis for a topology on a set \(X\), then the
topology generated by \(A\) equals the intersection of all topologies on
\(X\) that contain \(A\). Prove the same if \(A\) is a subbasis.

Lemma 13.1: The topology generated by \(A\) is the collection of all
unions of all elements of \(A\). 

\begin{proof}


Let the intersection of all topologies containing \(A\) be \(T_x =
\bigcap\limits_{\alpha \in J}T_{\alpha}\). Since they contain \(A\) and
are topologies, by property 2 of a topology, any \(T_{\alpha}\) must be
closed under arbitrary unions. Thus, \(T_A = \bigcup\limits_{\alpha \in
J} U_{\alpha}\), the topology generated by \(A\) (by lemma 13.1), is in
any of the above \(T_{\alpha}\). Therefore, \(T_A \subseteq T_x\).

Conversely, \(T_x\) is a topology, and closed under unions. Since \(T_A
\subseteq T_x\), then there is no set \(U \in T_x\) for which \(U \not\in
T_A\). Assume there is. By our construction of \(T_x\), this requires \(U\)
to be in every \(T_{\alpha}\) on X. However, \(T_A\) is a topology on \(X\)
which contains \(A\), and thus must be one of the
\(T_{\alpha}\). Therefore, \(U \in T_A\), a contradiction. Hence, \(\forall
U \in T_x, U \in T_A\), and \(T_x \subseteq T_A\). Therefore, \(T_x =
T_A\).
\end{proof}

Next, we prove that the topology generated by a subbasis \(A\) equals
the intersection of all topologies on \(X\) that contain \(A\). 

Definition: A \textit{subbasis} \(S\) for a topology on \(X\) is a
collection of subsets of \(X\) whose union equals \(X\). The
\textit{topology generated by $S$} is defined to be the collection \(T\)
of all unions of finite intersections of elements of \(S\).

\begin{proof}


Any topology that contains \(A\) must also contain the topology
generated by \(A\), since any topology must be closed under arbitrary
unions and finite intersections (properties 2 and 3 of
topologies). Thus, the intersection of all topologies on \(X\)
containing \(A\) must also include the topology generated by \(A\). That
the two are equal then follows from the subsetting arguments above.
\end{proof}
\end{document}