% Created 2020-03-24 Tue 17:36
% Intended LaTeX compiler: pdflatex
\documentclass[11pt]{article}
\usepackage[utf8]{inputenc}
\usepackage[T1]{fontenc}
\usepackage{graphicx}
\usepackage{grffile}
\usepackage{longtable}
\usepackage{wrapfig}
\usepackage{rotating}
\usepackage[normalem]{ulem}
\usepackage{amsmath}
\usepackage{textcomp}
\usepackage{amssymb}
\usepackage{capt-of}
\usepackage{hyperref}
\usepackage{amsthm}
\usepackage{amsmath}
\usepackage{amssymb}
\usepackage{graphicx}
\usepackage{fancyhdr}
\pagestyle{fancy}
\fancyhf{}
\rhead{Homework 6, Math 532}
\lhead{Jake Bailey}
\rfoot{Page \thepage}
\newtheorem{definition}{Definition}[section]
\author{Jacob Bailey}
\date{\today}
\title{Homework Set 6}
\hypersetup{
 pdfauthor={Jacob Bailey},
 pdftitle={Homework Set 6},
 pdfkeywords={},
 pdfsubject={},
 pdfcreator={Emacs 26.3 (Org mode 9.3.6)}, 
 pdflang={English}}
\begin{document}

\maketitle

\section{Problem 1}
\label{sec:orga73d30a}

Let \(f: X\rightarrow X\) be continuous. Show that if \(X = [0, 1]\), there is a
point \(x\in X\) such that \(f(x) = x\). The point \(x\) is called a fixed
point of \(f\). What happens if \(X\) is \([0,1)\) or \((0, 1]\)?


\begin{proof}
In all three cases considered, \(X\) is connected. By construction, \(f(0) \geq 0\)
and \(f(1) \leq 1\). Next, we construct a new function \(g(x) = f(x) - x\). This
function is also continuous, and we have that \(g(0) > 0\) and \(g(1) < 0\). By the
intermediate value theorem, we now have that \(\exists x \in (0,1)\) such that
\(g(x) = 0\), or \(f(x) = x\).
\end{proof}

If we remove the endpoints from \(X\), we are not guaranteed to find a
construction \(g\) which enables this proof. 
\section{Problem 2}
\label{sec:org12ca487}

Let \(X\) be an ordered set in the order topology. Show that if \(X\) is connected,
then \(X\) is a linear continuum. 


\begin{proof}
A linear continuum is an ordered set with two additional properties: the least
upper bound property, and that for any \(x,y\in X, x<y\), \(\exists z \in X\) such
that \(x < z < y\).

It is trivial that the second property holds for \(X\), for if it didn't, \(X\)
would contain a separation. 

To show the least upper bound property, we consider another set \(A\). If \(A\) is
non-empty and has an upper bound \(u\) but no least upper bound, consider the sets
\(A_1 = \{x\in X\ |\ \exists y(x) \in A : x \leq y(x)\}\) and \(A_2 = X\setminus
A_1\). As \(A \subset A_1\), it is non-empty. As \(u\) cannot be in \(A\) (as an upper
bound for \(A\) that is in \(A\) is a maximum which is also a least upper bound), \(u
> y\) for all \(y\in A\), so \(u \in A_2\), so \(A_2\) is non-empty as well. Both of
these sets are disjoint, and cover \(X\) by construction.

\(A_1\) is open, as for all \(x \in A_1\), \(x\in(-\infty, y(x)) \subset A_1\). If \(x
= y(x) \in A\), we know that some \(x'\in A\) must exist with \(x' > x\), as
otherwise \(x\) would be a maximum of \(A\), which we have assumed not to exist.
Then \(x \in (-\infty, x')\subset A_1\) as well. 

\(A_2\) is open, because if \(x \in A_2\), \(x\) is an upper bound for \(A\), but it
cannot be the smallest such (as \(A\) has been assumed to have no smallest upper
bound), so there is a smaller upper bound \(z\) for \(A\) and clearly \(x \in (z,
\infty) \subset A_2\). 

\(X\) having a cover by two open sets contradicts its connectedness, and so we
conclude that \(X\) must have a least upper bound. 

By both portions, we have shown that \(X\) satisfies the properties of a linear
continuum. \(X\) is a linear continuum. 
\end{proof}

\section{Problem 3}
\label{sec:org36f579f}

a) Is a product of path-connected spaces necessarily path connected? 
b) If \(A\subseteq X\) and \(A\) is path connected, is \(\overline{A}\) necessarily
path connected? 
c) If \(f: X\rightarrow Y\) is continuous and \(X\) is path connected, is \(f(X)\)
necessarily path connected? 
d) If \(\{A_{\alpha}\}\) is a collection of path-connected subspaces of \(X\) and if
\(\bigcap A_{\alpha} \not = \emptyset\) is \(\bigcup A_{\alpha}\) necessarily path
connected? 


a) 
\begin{proof}
Let \(x\) and \(y\) be two points in the product. We can construct a curve \(z(t)\)
between them as \(t \in [0,1]\), and each \(z_i(t)\) is a path connecting \(x_i\) and
\(y_i\) within \(X_i\). Since each of the \(X_i\) is path connected, these maps \(z_i\)
are continuous. Then, by theorem 19.6, the overall map \(z(t)\) is also
continuous. 
\end{proof}


b) No. A counterexample is the "Topologist's sine curve", the set of points \(T =
\{(x, sin(1/x)) | x \in (0,1]\}\cup\{(0,0)\}\).


c) Yes, the composition of two continuous functions is itself continuous, and
continuous maps preserve connectedness. 


d) 
\begin{proof}
Take any two points. If they are in the same set in the collection, there is a
path between them. If they are not in the same set in the collection then there
is a path connecting the first point to a common point of all sets in the
collection and another one connecting the common point to the second point, the
joint path is still continuous and is a path connecting the point.
\end{proof}
\section{Problem 4}
\label{sec:org43f65fb}

Show that every compact subspace of a metric space is bounded in that metric and
is closed. Find a metric space in which not every closed bounded subspace is
compact. 


\begin{proof}
Let \(S\) be a compact subspace of the metric space \(X\). Pick a point \(x\in S\).
Now, we consider the cover \(\{B(x,n)\ |\ n\in\mathbb{N}\}\), where \(B(x,n)\)
denotes the open ball centered at \(x\) of radius \(n\). We then simply note that we
will have a finite subcover, since \(S\) is compact. Choosing \(k\) as the largest
index to the cover, such that \(B(x,k)\) is the largest ball in the cover, we see
that \(S\) will be contained in \(B(x,k)\), and thus it is bounded. It must also be
closed: otherwise, \(B(x,k-1)\cup(B(x,k)\setminus S\) would be a cover of \(S\), and
it would not be compact. 
\end{proof}

As an example of a metric space in which not every closed bounded subspace is
compact, take \(\mathbb{R}\) with the discrete metric \(d(x,y) = 1\) iff \(x\not =
y\).
\section{Problem 5}
\label{sec:orgb3f60db}

Show that if \(f: X\rightarrow Y\) is continuous, where \(X\) is compact and \(Y\) is
Hausdorff, then \(f\) is a closed map (that is, \(f\) carries closed sets to closed
sets). 


\begin{proof}
Let \(A\) be closed in \(X\). Then \(A\), by theorem 26.2, \(A\) is compact. \(f\) is
continuous, so \(f(A)\) must be compact, since \(A\) is compact. Then, by theorem
26.3, \(f(A)\) is closed. Since \(A\) was arbitrary, this holds for any closed set
in \(X\). \(f\) is a closed map.
\end{proof}
\section{Problem 6}
\label{sec:orge072810}

Assume that \(\mathbb{R}\) is uncountable. Show that if \(A\) is a countable subset
of \(\mathbb{R}^2\), then \(\mathbb{R}^2 \setminus A\) is path connected. [Hint: How
many lines are there passing through a given point of \(\mathbb{R}^2\)?]


\begin{proof}
For any point there is an uncountable number of lines passing through it which
do not intersect A. For any two points there is a pair of lines that do
intersect each other but do not intersect the set A. So, both points are
connected to the point of intersection of the lines, and, therefore, are
connected.
\end{proof}
\end{document}
