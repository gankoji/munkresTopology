% Created 2020-04-17 Fri 14:55
% Intended LaTeX compiler: pdflatex
\documentclass[11pt]{article}
\usepackage[utf8]{inputenc}
\usepackage[T1]{fontenc}
\usepackage{graphicx}
\usepackage{grffile}
\usepackage{longtable}
\usepackage{wrapfig}
\usepackage{rotating}
\usepackage[normalem]{ulem}
\usepackage{amsmath}
\usepackage{textcomp}
\usepackage{amssymb}
\usepackage{capt-of}
\usepackage{hyperref}
\usepackage{amsthm}
\usepackage{amsmath}
\usepackage{amssymb}
\usepackage{graphicx}
\usepackage{fancyhdr}
\pagestyle{fancy}
\fancyhf{}
\rhead{Homework 8, Math 532}
\lhead{Jake Bailey}
\rfoot{Page \thepage}
\newtheorem{definition}{Definition}[section]
\author{Jacob Bailey}
\date{\today}
\title{Homework Set 8}
\hypersetup{
 pdfauthor={Jacob Bailey},
 pdftitle={Homework Set 8},
 pdfkeywords={},
 pdfsubject={},
 pdfcreator={Emacs 26.3 (Org mode 9.3.6)}, 
 pdflang={English}}
\begin{document}

\maketitle
\section{Problem 1}
\label{sec:org7a49c94}
Let \(X\) be limit point compact. 

a) If \(f: X\rightarrow Y\) is continuous, does it follow that \(f(X)\) is limit
point compact? 

No. If \(X\) is compact, the image is compact, so we need an example which is
limit point compact, but not compact. We take \(X\) as defined in Example 1 of
section 28, i.e. \(X = \mathbb{Z}_+\times Y\), where \(Y\) is a two point set with
the indiscrete topology. Then, the projection map in the first coordinate to
\(\mathbb{Z}_+\) is continuous, but maps the limit point compact space \(X\) to the
non-limi point compact space \(\mathbb{Z}_+\). 

b) If \(A\) is a closed subset of \(X\), does it follow that \(A\) is limit point
compact? 

Yes. An infinite subset of \(A\) has a limit point in \(X\) which is a limit point
of \(A\) as well. 

c) If \(X\) is a subspace of the Hausdorff space \(Z\), does it follow that \(X\) is
closed in \(Z\)? 

We comment that it is not generally true that the product of two limit point
copact spaces is itself limit point compact, even if the Hausdorff condition is
assumed. But the examples are fairly sophisticated. 

No. For a counterexample, we take the space from example 2: \(S_{\Omega}\subseteq
\overline{S}_{\Omega}\). \(S\) is not compact, but \(\overline{S}_{\Omega}\) is both
Hausdorff and compact, and thus limit point compact. Closedness would require
that \(S_{\Omega}\) also be compact, by theorem 26.3. 

\section{Problem 2}
\label{sec:org378247b}
A space \(X\) is said to be \textit{countably compact} if every countable open
covering of \(X\) contains a finite subcollection that covers \(X\). Show that for a
\(T_1\) space \(X\), countable compactness is equivalent to limit point compactness. 
[\textit{Hint:} If no finite subcollection of \(U_n\) covers \(X\), choose \(x_n
\not\in U_1\cup\ldots\cup U_n\) for each \(n\).]

\begin{proof}
Let \(X\) be a limit point compact \(T_1\) space and \(\{U_n\}\) be a countable open
covering of \(X\) such that there is no finite subcovering of \(X\). Let \(V_n =
U_1\cup\ldots\cup U_n\). Note that for every \(n\), \(V_n\) does not cover \(X\), but
for every \(x\in X\) there is a minimal \(n_x\) such that \(x \in V_{n_x}\). Let
\(x_0\in X\). For each \(n\geq 1\), let \(x_n\in X\setminus V_{n_{x_{n-1}}}\). This
defines an infinite subset of \(X\) that must have a limit point \(a\). But then the
neighborhood \(V_{n_a}\) of \(a\) contains only a finite number of elements in the
sequence, and for each of them that is not \(a\) we can find a neighborhood of \(a\)
that does not contain it (since \(X\) is \(T_1\)). The finite intersection of all
these neighborhoods with \(V_{n_a}\) is a neighborhood of \(a\) that does not
contain any point of the sequence which is not \(a\). This contradicts the fact
that \(a\) is a limit point of the sequence, and thus \(X\) must be countably
compact.

Next, suppose \(X\) is countably compact and \(Y\) is an infinite subset of \(X\).
There is a countably infinite subset \(Z\subseteq Y\) and every limit point of \(Z\)
is a limit point of \(Y\), as well. If no point in \(Z\) is a limit point, then
every point in \(Z\) has a neighborhood that does not contain any other points,
and the countable collection of such neighborhoods covers \(Z\). Since each set in
the collection covers one point of \(Z\) and \(Z\) is infinite, there is no finite
subcollection which also covers \(Z\). Therefore, \(Z\) is not closed (as a closed
subset of a countably compact space must also be countably compact). Thus, if
\(Z\) is closed, then it must have a limit point, and therefore be limit point
compact. 
\end{proof}
\section{Problem 3}
\label{sec:org70164db}
Let \(\{X_{\alpha}\}\) be an indexed family of nonempty spaces. 

a) Show that if \(\prod X_{\alpha}\) is locally compact, then each \(X_{\alpha}\) is
locally compact and \(X_{\alpha}\) is compact for all but finitely many values of
\(\alpha\). 

The projection map is continuous. Therefore, we may use the result of the next
exercise (exercise 3, page 186) to argue that all \(X_{\alpha}\) are compact. A
compact subspace of the product containing an open set has all but finitely many
projectionsequal to the whole corresponding space, since the projection is
continuous, these spaces must be compact. 

b) Prove the converse, assuming the Tychonoff theorem, that an arbitrary product
of compact spaces is compact. 

Assuming the Tychonoff theorem, all we need to prove is that the product of two
locally compact spaces is locally compact. For any product of two locally
compact spaces \(X\times Y\), we can simply find the corresponding compact subsets
and take their products. These product spaces are guaranteed to be compact, by
theorem 26.7. 

\section{Problem 4}
\label{sec:org7e398e4}
Prove that the one-point compactification of \(\mathbb{R}\) is homeomorphic with
the circle \(S^1\).

\begin{proof}
The circle without a single point is homeomorphic to the real line. Next, we use
two facts: first, that the one-point compactification of \(\mathbb{R}\) is unique
up to homeomorphism; second, that the compactification of the punctured circle
is the whole circle. Thus, the one-point compactification of \(\mathbb{R}\) is
homeomorphic to the circle. 
\end{proof}

\section{Problem 5}
\label{sec:org15f736d}
Let \(C\) be the Cantor set. 

a) Show that \(C\) is totally disconnected.

\begin{proof}
For any two points \(x,y\inC\), \(x\not = y\), there exists \(n\) such that they cannot
lie in the same closed interval \(A_n\). Therefore, there is a point \(z\) beteen
them which is not in \(C\), and the set can be separated by the rays \((-\infty,
z)\) and \((z, \infty)\).  
\end{proof}

b) Show that \(C\) is compact. 

\begin{proof}
\(C\) is a closed subspace of a compact space, \([0,1]\). Thus, by theorem 26.2, it
is compact. 
\end{proof}
\end{document}
