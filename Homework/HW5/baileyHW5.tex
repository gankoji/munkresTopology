% Created 2020-03-02 Mon 21:35
% Intended LaTeX compiler: pdflatex
\documentclass[11pt]{article}
\usepackage[utf8]{inputenc}
\usepackage[T1]{fontenc}
\usepackage{graphicx}
\usepackage{grffile}
\usepackage{longtable}
\usepackage{wrapfig}
\usepackage{rotating}
\usepackage[normalem]{ulem}
\usepackage{amsmath}
\usepackage{textcomp}
\usepackage{amssymb}
\usepackage{capt-of}
\usepackage{hyperref}
\usepackage{amsthm}
\usepackage{amsmath}
\usepackage{amssymb}
\usepackage{graphicx}
\usepackage{fancyhdr}
\pagestyle{fancy}
\fancyhf{}
\rhead{Homework 5, Math 532}
\lhead{Jake Bailey}
\rfoot{Page \thepage}
\newtheorem{definition}{Definition}[section]
\author{Jacob Bailey}
\date{\today}
\title{Homework Set 5}
\hypersetup{
 pdfauthor={Jacob Bailey},
 pdftitle={Homework Set 5},
 pdfkeywords={},
 pdfsubject={},
 pdfcreator={Emacs 28.0.50 (Org mode 9.3.6)}, 
 pdflang={English}}
\begin{document}

\maketitle

\section{Problem 1}
\label{sec:org4d2eab3}
Let \(\mathbb{R}^{\infty}\) be the subset of \(\mathbb{R}^{\omega}\) consisting of
all sequences that are "eventually zero," that is, all sequences \((x_1, x_2,
\ldots)\) such that \(x_i \not =0\) for only finitely many values of \(i\). What is
the closure of \(\mathbb{R}^{\infty}\) in \(\mathbb{R}^{\omega}\) in the box and
product topologies? Justify your answer. 

\section{Problem 2}
\label{sec:org53c449b}
Given sequences \((a_1, a_2, \ldots)\) and \((b_1, b_2, \ldots)\) of real numbers
with \(a_i > 0\) for all \(i\), define \(h:
\mathbb{R}^{\omega}\rightarrow\mathbb{R}^{\omega}\) by the equation

\begin{equation}
h((x\textsubscript{1}, x\textsubscript{2}, \ldots)) = (a\textsubscript{1x}\textsubscript{1} + b\textsubscript{1}, a\textsubscript{2x}\textsubscript{2} + b+2,\ldots).
\end{equation}

Show that if \(\mathbb{R}^{\omega}\) is given the product topology, \(h\) is a
homeomorphism of \(\mathbb{R}^{\omega}\) and itself. What happens if
\(\mathbb{R}^{\omega}\) is given the box topology? 

\begin{proof}
Let \(\mathbb{R}^{\omega}\) have the product topology, and \(h\) be as defined
aboved. If we know the sequences of \(a_i\) and \(b_i\), we can define a function
\(h^{-1}((x_1, x_2, \ldots)) = ((x_1 - b_1)/a_1, (x_2 - b_2)/a_2, \ldots)\). Since
all \(a_i\) are assumed to be greater than zero, and thus nonzero, \(h^{-1}\) is
obviously continuous. Since \(h\) is similarly continuous, we have that it is a
bijection with a continuous inverse between \(\mathbb{R}^{\omega}\) and itself,
i.e. a homeomorphism. 
\end{proof}

If we instead give \(\mathbb{R}^{\omega}\) the box topology, the result is
unchaged. 

\section{Problem 3}
\label{sec:orgb6f4138}
Prove that if \((X,d)\) is a metric space and \(X\) has the topology induced by \(d\),
then \(d: X\times X\rightarrow [0, \infty)\) is continuous, where \(X\times X\) has
the product topology. 

\begin{proof}
Let \(X\) be a metric space, with \(d\) a metric, and let \(X\times X\) have the
product topology. Call the interval \(A = [0, \infty)\), and let \(U\subseteq A\) be
an open subset. Then, \(d^{-1}(U)\) is of the form \(\bigcup\limits_{x\in X}
\{x\}\times V_x\), where \(V_x \subseteq X\) is a an open subset of \(X\) whose
points are less than \(a \in U\) (and greater than 0) in distance from the point
\(x\). The union of each of these \(V_x\) is obviously \(X\), as is the union of the
\(\{x\}\). \(X\times X\) is open in the product topology. We have considered all of
the open sets in \(A\), as the point \(d = 0\) cannot be in an open set (I've
assumed the standard topology in \(\mathbb{R}\)). \(d\) is continuous. 
\end{proof}

\section{Problem 4}
\label{sec:orgff2bc0a}

Show that \(\mathbb{R}_l\) and the ordered square satisfy the first countability
axiom. (This does not, of course, imply that they are metrizable). 

\begin{proof}
Let \(x\in\mathbb{R}_l\). Then, we construct a countable collection of
neighborhoods of \(x\) as \(V_n = [x - 1/n, x + 1/n)\), \(\forall n \in \mathbb{N}\).
It is clear from the axiom of completeness (eq. the nested interval principle,
the continuum hypothesis, etc) that for any neighborhood \(U\) of \(x\) we choose,
we can always find an \(n \in \mathbb{N}\) such that \(\{V_i\}_{i \geq n}\) are all
contained within \(U\). An almost identical argument holds for the ordered square. 
\end{proof}

\section{Problem 5}
\label{sec:org3410e70}
Show that the axiom of choice is equivalent to the statement that for any
indexed family \(\{A_{\alpha}\}_{\alpha \in J}\) of nonempty sets, with \(J \not =
\emptyset\), the Cartesian product \(\prod\limits_{\alpha\in J} A_{\alpha}\) is not
empty. 

\begin{proof}
First, we define the axiom of choice as the statement that "For every indexed
family \$$\backslash${A\textsubscript{\(\alpha\)}$\backslash$}\textsubscript{\(\alpha \in\) J} of nonempty sets, there exists an indexed
family \((x_{\alpha})_{\alpha \in J}\) of elements such that \(x_{\alpha}\in
A_{\alpha}\) for every \(\alpha \in J\)." 

Next, we let \(J\) be an index, such that \(J \not = \emptyset\). Consider the
Cartesian product \(\prod\limits_{\alpha\in J} A_{\alpha}\). By the axiom of
choice above, we have that there exists a set of elements
\((x_{\alpha})_{\alpha\in J}\), such that \(x_{\alpha}\in A_{\alpha}\) for each
\(\alpha\). Thus, the Cartesian product cannot be empty. 
\end{proof}

\textit{Note:} The definition I use above for the axiom of choice can be found
at \url{https://en.wikipedia.org/wiki/Axiom\_of\_choice}. It was significantly easier to
use here than the version presented in Munkres' text. 
\end{document}
