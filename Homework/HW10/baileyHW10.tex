% Created 2020-05-02 Sat 15:27
% Intended LaTeX compiler: pdflatex
\documentclass[11pt]{article}
\usepackage[utf8]{inputenc}
\usepackage[T1]{fontenc}
\usepackage{graphicx}
\usepackage{grffile}
\usepackage{longtable}
\usepackage{wrapfig}
\usepackage{rotating}
\usepackage[normalem]{ulem}
\usepackage{amsmath}
\usepackage{textcomp}
\usepackage{amssymb}
\usepackage{capt-of}
\usepackage{hyperref}
\usepackage{amsthm}
\usepackage{amsmath}
\usepackage{amssymb}
\usepackage{graphicx}
\usepackage{fancyhdr}
\pagestyle{fancy}
\fancyhf{}
\rhead{Homework 10, Math 532}
\lhead{Jake Bailey}
\rfoot{Page \thepage}
\newtheorem{definition}{Definition}[section]
\author{Jacob Bailey}
\date{\today}
\title{Homework Set 10}
\hypersetup{
 pdfauthor={Jacob Bailey},
 pdftitle={Homework Set 10},
 pdfkeywords={},
 pdfsubject={},
 pdfcreator={Emacs 26.3 (Org mode 9.3.6)}, 
 pdflang={English}}
\begin{document}

\maketitle
\section{Problem 1}
\label{sec:orgf0c4a9f}
Show that if \(h, h': X\rightarrow Y\) are homotopic and \(k, k':Y\rightarrow Z\)
are homotopic, then \(k\circ h\) and \(k'\circ h'\) are homotopic.

There exist two homotopies \(H: X\times I\rightarrow Y\) between \(h\) and \(h'\), and
\(K:Y\times I\rightarrow Z\) between \(k\) and \(k'\). Consider the map \(F: X\times
I\rightarrow Z\) such that \(F(x,t) = K(H(x,t),t)\). Clearly, this is a homotopy
between \(k\circ h\) and \(k'\circ h'\). 

\section{Problem 2}
\label{sec:orgb931fb8}
Let \(\alpha\) be a path in \(X\) from \(x_0\) to \(x_1\); let \(\beta\) be a path in \(X\)
from \(x_1\) to \(x_2\). Show that if \(\gamma = \alpha * \beta\), then \(\hat{\gamma}
= \hat{\alpha}*\hat{\beta}\).

$$\hat{\gamma}([f]) = [\overline{\alpha *
\beta}]*[f]*[\overline{\beta}*\overline{\alpha}]$$

$$ = [\overline{\beta}*\overline{\alpha}]*[f]*[\alpha * \beta]$$

$$ = [\overline{\beta}]*[\overline{\alpha}]*[f]*[\alpha]*[\beta]$$

$$ = [\overline{\beta}]*\hat{\alpha}([f])*[\beta]$$

$$ = \hat{\beta}\circ \hat{\alpha}([f]).$$

\section{Problem 3}
\label{sec:org949202d}
Let \(x_0\) and \(x_1\) be points of the path-connected space \(X\). Show that
\(\pi_1(X, x_0)\) is abelian if and only if for every pair \(\alpha\) and \(\beta\) of
paths from \(x_0\) to \(x_1\), we have \(\hat{\alpha} = \hat{\beta}\).

For every \(\alpha, \beta, \hat{\alpha} = \hat{\beta}\) iplies that for every
\(\alpha, \beta, f\), a loop based at \(x_0, \hat{\alpha([f])} =
\hat{\beta([f])}\). Further, this implies that
\([f]\circ[\alpha\circ\beta] = [\alpha\circ\beta]\circ[f]\). 

Note that \(\alpha\circ\beta\) is an arbitrary loop based at \(x_0\) and passing
through \(x_1\). So, if the group of the path homotopy classes of loops based at
\(x_0\) is Abelian, then the right hand side expression holds for arbitrary
\(\alpha, \beta\), and \(f\), therefore for every \(\alpha, \beta, \hat{\alpha} =
\hat{\beta}\). 

Vice versa, if for every \(\alpha, \beta, \hat{\alpha} = \hat{\beta}\), then we
have shown that the group is commutative when at least one of the terms is a
path homotopy class of a loop passing through \(x_1\). So, take arbitrary \([f],
[g] \in \pi_1(X, x_0)\) and take any path \(\alpha\) from \(x_0\) to \(x_1\). Then,
\(g\circ\alpha\circ\overline{\alpha}\) is a loop based at \(x_0\) passing through
\(x_1\). Then, \([f]\circ[g\circ\alpha\circ\overline{\alpha}] = 
[g\circ\alpha\circ\overline{\alpha}]\circ[f]\). 

\section{Problem 4}
\label{sec:org547a0fe}
Show that the mapa of Example 3 on page 338 is a covering map. Generalize to the
map \(p(z) = z^n\).

Example 3: \(p: S^1\rightarrow S^1\) given by \(p(z) = z^2\) where \(S^1\) is
considered as a subspace of the complex plane. 

Consider a general map \(p(z) = z^n\). If \(U = \{z = e^{i\phi}\ |\ \phi\in
(a,b)\subseteq (0, 2\pi)\}\). Then, \(p^{-1}(U) = \{z = e^{i\phi}|\ \phi\in (a/n +
2\pi k/n, b/n + 2\pi k/n), k\in \overline{0, n-1}\}\) iis the union of \(n\) open
intervals in \(S^1\) such that the restriction of \(p\) onto each such interval is a
homeomorphism of the interval with \(U\). This is also the case if we restrict
\(\phi\) to \((-\pi, \pi)\). Overall, ever point of \(s^1\) has such an open
neighborhood. 

\section{Problem 5}
\label{sec:org284b307}
Show that if \(X\) is path connected, the homomorphism induced by a continuous map
is independent of base point, up to isomorphism of the groups involved. More
precisely, let \(h: X\rightarrow Y\) be continuous, with \(h(x_0) = y_0\) and
\(h(x_1) = y_1\). Let \(\alpha\) be a path in \(X\) from \(x_0\) to \(x_1\), and let
\(\beta = h\circ \alpha\). Show that 

\(\hat{\beta}\circ(h_{x_0})_* = (h_{x_1})_*\circ \hat{\alpha}\). 

This equation expresses the fact that the following diagram of maps "commutes."

(See graph in text. That's a bit beyond my own \TeX{}-ing ability, at the moment.)


$$ \hat{\beta}\circ(h_{x_0})([f])
= [\overline{\beta}]*(h_{x_0})([f])*[\beta] $$

$$ = [h\circ \overline{\alpha}]*[h\circ f]*[h\circ\alpha] $$
$$ = [h\circ(\overline{\alpha}*f*\alpha)] $$ 
$$ = (h_{x_1})[\overline{\alpha}*f*\alpha] $$ 
$$ = (h_{x_1})\circ \alpha([f]) $$
\end{document}
