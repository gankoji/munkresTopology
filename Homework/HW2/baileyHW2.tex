% Created 2020-01-28 Tue 20:47
% Intended LaTeX compiler: pdflatex
\documentclass[11pt]{article}
\usepackage[utf8]{inputenc}
\usepackage[T1]{fontenc}
\usepackage{graphicx}
\usepackage{grffile}
\usepackage{longtable}
\usepackage{wrapfig}
\usepackage{rotating}
\usepackage[normalem]{ulem}
\usepackage{amsmath}
\usepackage{textcomp}
\usepackage{amssymb}
\usepackage{capt-of}
\usepackage{hyperref}
\usepackage{amsthm}
\usepackage{amsmath}
\usepackage{amssymb}
\usepackage{graphicx}
\usepackage{fancyhdr}
\pagestyle{fancy}
\fancyhf{}
\rhead{Homework 2, Math 532}
\lhead{Jake Bailey}
\rfoot{Page \thepage}
\author{Jacob Bailey}
\date{\today}
\title{Homework Set 2}
\hypersetup{
 pdfauthor={Jacob Bailey},
 pdftitle={Homework Set 2},
 pdfkeywords={},
 pdfsubject={},
 pdfcreator={Emacs 26.3 (Org mode 9.3.1)}, 
 pdflang={English}}
\begin{document}

\maketitle

\section{Problem 1}
\label{sec:orgcb6239d}

If \(L\) is a straight line in the plane, describe the topology \(L\)
inherits as a subspace of \(\mathbb{R}_l\times\mathbb{R}\) and as a subspace of
\(\mathbb{R}_l\times\mathbb{R}_l\). In each case, it is a familiar topology.

To be precise, we define a line \(L\) as a set of points \(L = \{ x,y \in
\mathbb{R}^2 |\ y = mx + b, m,b \in \mathbb{R}\}\). In both cases, the
basis elements for the subspace topology \(L\) inherits can be defined
as \(\{ B_l\times B \cap L \}\).

As a subspace of \(\mathbb{R}_l\times\mathbb{R}\), we have basis
elements of the form \([x_1,x_2)\times(y_1,y_2)\cap L\). Since a line is
one dimensional, we can parameterize \(x,y\) as functions of a dummy
\(t\), such that \(y(t) = mx(t) + b\), and we are left with a one
dimensional set with the lower limit topology, i.e. \(\mathbb{R}_l\).

The same holds for \(L\) as a subspace of
\(\mathbb{R}_l\times\mathbb{R}_l\).

\section{Problem 2}
\label{sec:org40d59ba}

Let \(I = [0,1]\). Compare the product topology on \(I\times I\), the
dictionary order topology on \(I\times I\), and the topology \(I\times I\)
inherits as a subspace of \(\mathbb{R}\times\mathbb{R}\) in the dictionary order
topology.

The product topology on \(I\times I\) is finer than the the dictionary
order topology on \(I\times I\), which is both finer and coarser than
the subspace topology inherited from \(\mathbb{R}\times\mathbb{R}\) in
the dictionary order topology.

\begin{proof}


\textit{Product Topology is finer than the Dictionary Order Topology:}
The basis for the product topology is sets of the form
\(((a,b),(c,d))\), \(a,b,c,d \in I\). Basis elements of the dictionary
order topology are also intervals, with endpoints \((p_1,p_2)\) and
\((q_1,q_2)\): \(B = \{ (a,b), a,b \in I\ |\ (p_1 < a \leq q_1)\
\text{or}\ (p_1 = a)\ \text{and}\ (p_2 < b \leq q_2)\}\).

Clearly, since both elements are drawn from the same set \(I\), we can
find a slice \(B\) in the dictionary order topology which contains any
given element \(((a,b),(c,d))\) of the product topology, since elements
from the product topology do not contain their endpoints. The product
topology is finer than the dictionary order topology.

\textit{Subspace topology is equal to the DOT on $I\times I$:} This
follows from the definition of the subspace topology and the
definition of the DOT on \(I\times I\). First, the subspace topology:

The subspace topology on \(I\times I\) is that topology having as basis
the collection of sets \(D_I = \{ d\cap I | d \in D\}\) where \(D\) is the
basis of the DOT in \(\mathbb{R}^2\). These are intervals \((p,q) \in I\)
with the dictionary ordering. The basis of the DOT on \(I\times I\) is
exactly this collection of sets, as well. The two are equal, and thus
comparable.
\end{proof}

\section{Problem 3}
\label{sec:org2b55c60}

Let \(Y\) be a subspace of \(X\). Prove that if \(A\) is closed in \(Y\) and
\(Y\) is closed in \(X\), then \(A\) is closed in \(X\).

\begin{proof}


Let \(A\) be closed in \(Y\). Then its complement, \(A_C = Y \setminus A\),
is open in \(Y\) (and thus an element of the topology on \(Y\)). Since \(Y\)
is a subspace of \(X\), \(A_C\) is also an element in the topology of
\(X\). Topologies are closed under arbitrary unions, so the set \(A_{CX}
= A_C \cup (Y\setminus X)\) is also in the topology. Thus, the
complement of \(A\) in \(X\) is open, and therefore \(A\) is closed in \(X\).
\end{proof}

\section{Problem 4}
\label{sec:org8c337ce}

Show that if \(A\) is closed in \(X\) and \(B\) is closed in \(Y\), then
\(A\times B\) is closed in \(X\times Y\).

\begin{proof}


Let \(A\) be closed in \(X\), and \(B\) be closed in \(Y\). Then, \(A_C =
A\setminus X\) is an element of the topology on \(X\), and \(B_C =
B\setminus Y\) is an element of the topology on \(Y\). Since the product
topology on \(X\times Y\) is inherited from the sets \(X\) and \(Y\), the
set \(A_C \times B_C\) is in this product topology. This set is also the
complement of \(A \times B\) in \(X\times Y\). Thus, \(A\times B\) is closed
in \(X\times Y\).
\end{proof}

\section{Problem 5}
\label{sec:org707405d}

Show that the dictionary order topology on the set
\(\mathbb{R}\times\mathbb{R}\) is the same as the product topology
\(\mathbb{R}_d\times\mathbb{R}\) where \(\mathbb{R}_d\) denotes
\(\mathbb{R}\) in the discrete topology. Compare this topology with the
standard topology on \(\mathbb{R}^2\).

\begin{proof}


First, recall that a basis element in the dictionary order topology on
\(\mathbb{R}\times\mathbb{R}\) is an interval of the form \((p,q), p,q \in
\mathbb{R}\times\mathbb{R}\).

Next, we note from theorem 15.1 we have that the basis for the product topology
on \(\mathbb{R}_d\times\mathbb{R}\) is the collection of sets \(D = \{ b\times
c\ |\ b\in B, c\in C\}\), where \(B,C\) are the bases of the discrete and standard
topologies on \(\mathbb{R}\), respectively.

To the proof. Let \(A = ({x},(a,b))\) be a basis element of
\(\mathbb{R}_d\times\mathbb{R}\). Choose \(p_1, p_2 \in \mathbb{R}\times\mathbb{R},
p_1 = (a_1,b_1), p_2 = (a_2,b_2)\). Then, let \(a_1 = a_2 = x\). Clearly, then, the
interval \((p_1,p_2) = ({a_1},(b_1,b_2))\), which is equivalent to our chosen \(A\)
from the product topology. Thus, for any element of the topology on
\(\mathbb{R}_d\times\mathbb{R}\), we can find an equivalent element in
\(\mathbb{R}\times\mathbb{R}_{DOT}\).
\(\mathbb{R}_d\times\mathbb{R}\subseteq\mathbb{R}\times\mathbb{R}_{DOT}\).

Next, let \(p = (a,b)\) and \(q = (c,d)\). Then, we can choose elements of the basis
of \(\mathbb{R}_d\times\mathbb{R}\) as \(r_i = ({p + i\epsilon}, (b,d))\). By
construction, \(r_i \in C\), the basis on \(\mathbb{R}_d\times\mathbb{R}\). Let \(R =
\bigcup\limits_{i \in J}r_i = (p,q)\). Topologies are closed under arbitrary
unions, so \(R \in C\) as well. But \(R = (p,q)\), so \((p,q) \in
\mathbb{R}_d\times\mathbb{R}\), and
\(\mathbb{R}\times\mathbb{R}_{DOT}\subseteq\mathbb{R}_d\times\mathbb{R}\).
Therefore, \(\mathbb{R}\times\mathbb{R}_{DOT} = \mathbb{R}_d\times\mathbb{R}\).
\end{proof}

Next, we prove that \(\mathbb{R}_d\times\mathbb{R}\) is strictly finer than
\(\mathbb{R}^2_{std}\).

\begin{proof}


Let \(A = ({x},(a,b))\) be a basis element of \(\mathbb{R}_d\times\mathbb{R}\).
Choose \(c < x < d, c,d \in\mathbb{R}\), and let \(S = ((c,d),(a,b))\). Clearly, \(S
\in \mathbb{R}^2_{std}\), by construction, and also \(A \subset S\).
\(\mathbb{R}_d\times\mathbb{R}\) is finer than \(\mathbb{R}^2_{std}\).

However, we cannot go the other way. Topologies are only closed under finite
intersections, and we cannot make the set \(A\) by any finite number of
intersections of sets of the form of \(S\), due to the completeness of the reals
(equivalently, the nested interval principle).

Thus, \(\mathbb{R}_d\times\mathbb{R}\) is strictly finer than
\(\mathbb{R}^2_{std}\).
\end{proof}
\end{document}