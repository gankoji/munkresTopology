% Created 2020-05-13 Wed 18:53
% Intended LaTeX compiler: pdflatex
\documentclass[11pt]{article}
\usepackage[utf8]{inputenc}
\usepackage[T1]{fontenc}
\usepackage{graphicx}
\usepackage{grffile}
\usepackage{longtable}
\usepackage{wrapfig}
\usepackage{rotating}
\usepackage[normalem]{ulem}
\usepackage{amsmath}
\usepackage{textcomp}
\usepackage{amssymb}
\usepackage{capt-of}
\usepackage{hyperref}
\usepackage{amsthm}
\usepackage{amsmath}
\usepackage{amssymb}
\usepackage{graphicx}
\usepackage{fancyhdr}
\pagestyle{fancy}
\fancyhf{}
\rhead{Final Exam, Math 532}
\lhead{Jake Bailey}
\rfoot{Page \thepage}
\newtheorem{definition}{Definition}[section]
\author{Jacob Bailey}
\date{\today}
\title{Final Exam}
\hypersetup{
 pdfauthor={Jacob Bailey},
 pdftitle={Final Exam},
 pdfkeywords={},
 pdfsubject={},
 pdfcreator={Emacs 26.3 (Org mode 9.3.6)}, 
 pdflang={English}}
\begin{document}

\maketitle
\section{Problem 1}
\label{sec:org1db0e48}

Show that the space \(\mathbb{R}_l\) is not connected. 

Take two sets, \((-\infty, 0)\) and \([0, \infty)\). The union of these two sets is
indeed \(\mathbb{R}_l\). Further, both of these sets are open in \(\mathbb{R}_l\).
Thus, together they form a separation in \(\mathbb{R}_l\), and the space is
therefore not connected. 

\section{Problem 2}
\label{sec:org382666b}

Let \(X\) be a topological space and let \(Y\) be its one-point compactification.
Show that if \(X\) is connected and not compact, then \(Y\) must be connected. 

By definition, a one-point compactification is a compact Hausdorff space, and
the difference of the compactification and the original set is a singe point.
Assume that \(X\) is connected and not compact. Further, assume that there exists
a separation in \(Y\), i.e. there exist two disjoint open sets, \(U\) and \(V\), whose
union equals \(Y\). One of these sets must contain the point \(a = Y\setminus X\),
let's say \(V\) does. Then, \(U\subset X\), and \(W = V\setminus\{a\} \subset X\).
Since \(Y\) is the one-point compactification of \(X\), we also have that \(U\cup W =
X\). By the Hausdorff nature of \(Y\), we know that \(\{a\}\) is closed, and thus \(W\) is
open. Thus, we have shown that if \(Y\) contains a separation, so does \(X\), and
\(X\) is not connected, a contradiction. \(Y\) is connected.  

\section{Problem 3}
\label{sec:org0f01fd7}

Let \(X\) be an uncountable set with the countable complement topology. 

a) Is \(X\) connected? Why or why not?

Yes. There cannot exist two disjoint open sets whose union equals \(X\). Any open
set which is not \(\emptyset\) or \(X\) must have a countable complement, which by
definition of the CCT is closed.  

b) Does \(X\) have the \(T_1\) separation property? Why or why not?

Yes, because finite point sets are countable, and thus closed in the CCT. 

c) Is \(X\) Hausdorff? Why or why not?

No. There cannot exist two disjoint open sets which contain, respectively, two
arbitrary points separately.  

d) Is \(X\) metrizable? Why or why not?

No, because it is not regular. Theorem 40.3 (The Nagata-Smirnov metrization
theorem) states that a space is metrizable if and only if it is both regular and
has a basis that is countably locally finite. 

e) Let \(a \in X\) and define \(A = X\setminus \{a\}\). Show that \(a
\in\overline{A}\).

If \(A\) is as defined above, then the smallest closed set that contains it is \(X\).
\(a\in X\), thus it is in the closure.  

f) Let \(x_1, x_2, x_3,\ldots\) be a sequence in \(X\) such that each \(x_i\in A\).
Show that the sequence does \textit{not} converge to \(a\).

I actually don't see a reason for this to be true. In the CCT, open sets (and
thus neighborhoods) are huge, uncountably so. There are no small neighborhoods
of the point \(a\), and the space isn't Hausdorff, so why would an arbitrary
sequence not be able to converge to \(a\), even if it's not in the set \(A\)? It's
in the closure of \(a\), so it must be a limit point, and thus every neighborhood
of it must intersect \(A\) at some point other than \(a\). If our sequence arrives,
at some integer \(n\), at that point, and remains there or gets closer to \(a\) from
then on, wouldn't the sequence have converged to \(a\)?  

g) Is \(X\) first countable? Why or why not? 

Yes, because for each point \(x\in X\), there exist at most countably many
neighborhoods, and thus there is a countable collection of those neighborhoods
which are all contained within the neighborhoods of \(x\). 


\section{Problem 4}
\label{sec:org0df1a95}

Prove that if \(X\) is a Lindel $\backslash$\"of space, then every uncountable subset of \(X\) has a limit point. 

A Lindel $\backslash$\"of space is one for which every open covering contains a countable
subcovering. A limit point of a set \(A\) is one for which every neighborhood of
said point intersects \(A\) in some point other than itself. 

\begin{proof}
Let \(A\subseteq X\) be uncountable, and assume that \(A\) has no limit points.
Then, every point in \(A\) must be isolated. In order to cover this set, we would
then need uncountably many open sets, for which no finite subcover exists. This
is a contradiction of the assumption that the space \(X\) is Lindel$\backslash$\"of, and thus
\(A\) must have at least one limit point. 
\end{proof}

\section{Problem 5}
\label{sec:org7465c66}

Let \(A\subseteq X\). Suppose \(r: X\rightarrow A\) is a continuous map such that
\(r(a) = a\) for each \(a\in A\). If \(a_0 \in A\), show that

\(r_*: \pi_1(X, a_0)\rightarrow \pi_1(A, a_0)\)

is surjective.

Essentially, the task here is to show that the continuous map \(r\), which is the
identity for the subset \(A\) of the space \(X\), preserves the fundamental group of
\(X\) at the point \(a_0\). As defined on page 333 of Munkres, the map \(r_*\) is a
homomorphism induced by the map \(r\). Furthermore, from corollary 52.5, if \(r\)
is a homeomorphism of \(X\) with \(A\) (trivially, it is), then \(r_*\), the
homomorphism induced by it, is an isomorphism. Hence, \(r_*\) is surjective.   

\section{Problem 6}
\label{sec:orgc2c198d}

Show that any covering map \(p: E\rightarrow B\) is an open map.

By definition, a covering map \(p\) is continuous, surjective, and for any open
set \(U \subseteq B\), the pre-image can be written as a union of disjoint open
sets in \(E\), where the restriction of \(p\) to these disjoint open sets is a
homeomorphism. 

Since \(p\) is a homeomorphism under these restrictions, it is a continuous
bijection in both directions. Thus, \(p^{-1}(U)\) is open, as well as
\(p(V_{\alpha})\). Since \(p\) carries open sets to open sets, it is an open map. 
\end{document}
