% Created 2020-04-08 Wed 16:56
% Intended LaTeX compiler: pdflatex
\documentclass[11pt]{article}
\usepackage[utf8]{inputenc}
\usepackage[T1]{fontenc}
\usepackage{graphicx}
\usepackage{grffile}
\usepackage{longtable}
\usepackage{wrapfig}
\usepackage{rotating}
\usepackage[normalem]{ulem}
\usepackage{amsmath}
\usepackage{textcomp}
\usepackage{amssymb}
\usepackage{capt-of}
\usepackage{hyperref}
\usepackage{amsthm}
\usepackage{amsmath}
\usepackage{amssymb}
\usepackage{graphicx}
\usepackage{fancyhdr}
\pagestyle{fancy}
\fancyhf{}
\rhead{Exam 2, Math 532}
\lhead{Jake Bailey}
\rfoot{Page \thepage}
\newtheorem{definition}{Definition}[section]
\author{Jacob Bailey}
\date{\today}
\title{Exam 2}
\hypersetup{
 pdfauthor={Jacob Bailey},
 pdftitle={Exam 2},
 pdfkeywords={},
 pdfsubject={},
 pdfcreator={Emacs 26.3 (Org mode 9.3.6)}, 
 pdflang={English}}
\begin{document}

\maketitle
\section{Problem 1}
\label{sec:org28fe9cb}
Let \(Y_1\) and \(Y_2\) be compact subspaces of a topological space \(X\). 

a) Prove that the union \(Y_1\cup Y_2\) is compact if both \(Y_1\) and \(Y_2\) are
compact. Give an example to show that the converse is false. 

\begin{proof}
Let \(Y_1\) and \(Y_2\) be compact. By definition, any covering of each of these
sets has a finite subcover. Let \(A_1\) and \(A_2\) be the corresponding finite
subcovers for arbitrary covers of each \(Y\). Then, \(A = A_1\cup A_2\) covers
\(Y_1\cup Y_2\). Since the union of two finite sets is itself finite, \(A\) is a
finite cover of \(Y_1\cup Y_2\), and thus the union is compact. 
\end{proof}

b) Does the result of part (a) together with the induction principle imply that
a \textit{finite} union of compact subspaces is compact? Explain why, or find a
counterexample. 

Yes, because the union of finitely many finite sets is still itself finite. 

c) Does the result of part (a) together with the induction principle imply that
a \textit{countable} union of compact subspaces is compact? Explain why, or find
a counter example. 

No, because a countable union of finite sets could itself no longer be finite. A
counterexample would be a collection of compact subspaces of \(\mathbb{R}\) of the
form \([a, a+1], a\in\mathbb{Z}\). The countable union of these sets forms
\(\mathbb{R}\), which is known to not be compact.  

\section{Problem 2}
\label{sec:org21f7b82}
Let \(X_3 = \{a, b, c\}\) be a topological space containing exactly three points,
and let \(X_4 = \{a, b, c, d\}\) be a topological space containing exactly four
points. 

a) Prove that if there is no open singleton in \(X_3\), then \(X_3\) is connected.

\begin{proof}
By assumption, no subset of \(X_3\) of the form \(\{x\}\) is open. Thus, our only
options for open sets are the empty set, \(X_3\) itself, and sets of the form
\(\{x_1, x_2\}\). Clearly, no combination of two \textit{distinct} sets of this
form and/or the emptyset and/or \(X_3\) can contain all the points of \(X_3\), so
\(X_3\) must be connected. 
\end{proof}

b) Is the following true? If \(X_3\) is connected, then there is no open singleton
in \(X_3\).

If \(X_3\) is connected, there exists no possible separation (two distinct open
sets, whose intersection is the empty set) in \(X_3\). The only way to construct
such a separation is with singletons, due to the three pointed nature of the
set.  

c) Is the following true? If there is no open singleton in \(X_4\), then \(X_4\) is
connected. 

No. \(\{a,b\}\) and \(\{c, d\}\) are a separation in \(X_4\) that does not require
singletons for construction. \(X_4\) is not connected. 

\section{Problem 3}
\label{sec:org4839279}
Let \(X = \mathbb{R}^{\omega}\) with the product topology. Let \(A\subseteq X\) be
the subset \(A = [0,1]\times [0,2]\times [0,3]\times\ldots\) endowed with the
subspace topology. Explain each answer below. 

a) Is \(X\) metrizable? 

Yes. See theorem 20.5, page 125 of Munkres. 

b) Is \(X\) compact? 

No. None of its constituents (\(\mathbb{R}\)) is compact, so
the product cannot be either. 

c) Is \(X\) sequentially compact? If not, construct a sequence in \(X\) that has no
convergent subsequence. 

No, since it is metrizable but not compact. The sequence \(\{0, \ldots x_i,
\ldots 0 \ldots\}_{i\in\mathbb{Z}_+}\) for a given nonzero point \(x\) seems to fit
the bill.

d) Is \(X\) connected? 

Yes, since its constituents are all connected. A path function on each
\(\mathbb{R}\) can be found that is continuous, and the product of such continuous
functions is continuous (theorem 19.6), so actually it's path connected.

e) Is \(A\) metrizable?

Yes. The Euclidean metric applies equally well to subspaces of \(\mathbb{R}\) as
it does to all of \(\mathbb{R}\). 

f) A is compact by the Tychonoff theorem. Is \(A\) sequentially compact? If not,
construct a sequence in \(X\) that has no convergent subsequence. 

\(A\) is sequentially compact, since it is compact and metrizable, by theorem
28.2.

g) Is \(A\) connected? 

Yes, by a similar argument to the above for all of \(X\). That each constituent is
a single connected interval is the main point. 


\section{Problem 4}
\label{sec:org7db0c63}
Let \(X\) be \(\mathbb{R}\) with the countable complement topology. 

a) Find the closure and the interior of \(Y = (0,\infty)\subseteq X\). 

The closure is the smallest closed set which contains \(Y\). Since \([0,
\infty)\) is not countable, it cannot be the complement of an open set (and thus
not closed). Since it is uncountable, the next largest closed set (as long as
we're subscribing to the continuum hypothesis, anyway) is \(\mathbb{R}\).
\(\overline{Y} = \mathbb{R}\). 

Next, the interior. The is the largest open set which is contained within \(Y\).
Since any open set in the countable complement topology must either be the
emptyset or \(\mathbb{R}\) minus some countable set, we conclude that the largest
open set which is completely contained within \(Y\) is the empty set. \(\text{int}\
Y = \emptyset\).

b) Consider the map \(f: X\rightarrow X\) defined by \(f(x) = cos(x)\). Is \(f\)
continuous on \(X\) with the countable complement topology? (\textit{Hint:} It may
be helpful to know that the countable union of countable sets is countable.)

Indeed, \(f\) is continuous, albeit vacuously. The range of \(f\) is actually \([-1,
1]\), whose complement is obviously not countable, and the complement of any
subset of it is also not countable. Thus, no subset of the range of \(f\) is open.
Therefore, vacuously, for every open set  \(A \subseteq [-1, 1]\), \(f^{-1}(A)\) is
also open. (Actually, \(\emptyset\) is open, but its preimage, also the empty set,
is open). 

c) Is \(X\) compact? (\textit{Hint:} Consider the cover
\(\{U_k\}_{k\in\mathbb{Z}_+}\) where \(U_k =
(\mathbb{R}\setminus\mathbb{Z}_+)\cup\{k\}\).)

No. A finite subcover must contain finitely many \textit{finite sets}. No set
which is open in this topology (aside from the empty set, which by definition
covers nothing) is finite.

d) Is \(X\) connected? 

Yes. We require two \textit{distinct} sets (i.e \(A\cap B = \emptyset\)) whose
union covers the space for a separation. 

If we consider the complements, \((A\cap B)^c = \mathbb{R} = A^c\cup B^c\) by
DeMorgan's laws, we see that this would require that we be able to form
\(\mathbb{R}\) from the union of two countable sets (complements must be countable
for \(A\) and \(B\) to be open). The union of countably many (2 is countable!)
countable sets is itself countable, while \(\mathbb{R}\) is uncountable.


\section{Problem 5}
\label{sec:org4670bbe}

Let \(X\) be a Hausdorff space, let \(C\) be a compact subset of \(X\), and let \(a\) be
a point of \(X\) which is not in \(C\). Prove that there are disjoint open sets \(U\)
and \(V\) with \(C\subseteq U\) and \(a\in V\).

This is Lemma 26.4 of Munkres, on page 166. Proof is found on the preceding
page, 165, as part of the proof of theorem 26.3. Below I present an excerpt from
that proof that deals just with this lemma. 

\begin{proof}
Let \$X,C,\$ and \(a\) be as above. That is, \(a\) is in \(X\setminus C\). For each
point \(c\in C\), let us choose disjoint neighborhoods \(U_c\) and \(V_c\) of the
points \(a\) and \(c\), respectively. We know these exist since \(X\) is Hausdorff. 

The collection \(\{V_c\ |\ c\in C\}\) is a covering of \(C\) by sets open in \(X\).
Since \(C\) is compact, we know that finitely many of the \(V_c\) cover \(C\). Thus,
we have that the open set \(V = V_{c_1}\cup \ldots \cup V_{c_n}\) contains \(C\),
and it is disjoint from the open set \(U = U_{c_1}\cap \ldots \cap U_{c_n}\)
formed by taking the intersection of the corresponding neighborhoods of \(a\).
For if \(z\) is a point of \(V\), then \(z\in V_{c_i}\) for some \(i\), and hence
\(z\not\in U_{c_i}\) by construction, and then \(z\not\in U\). 

Thus, \(U\cap V = \emptyset\), \(a\in U\), and \(Y\subseteq V\). 
\end{proof}
\end{document}
